\begin{question}
    {6.2.10}
    {
        Use an element argument to prove the statement.
        Assume that all sets are subsets of a universal set $U$.
        For all sets $A, B$, and $C$,
        \begin{align*}
            (A-B) \cap(C-B)=(A \cap C)-B \text{. }
        \end{align*}
        \vspace{-\baselineskip}
    }
\end{question}
\begin{proof}
    Consider arbitrary sets \(A, B, C\). The given statement is proven true by establishing two key sub-statements:
    \begin{enumerate}
        \item[1.] \((A - B) \cap (C - B) \subseteq (A \cap C) - B\)
        \item[2.] \((A \cap C) - B \subseteq (A - B) \cap (C - B)\)
    \end{enumerate}
    Firstly, for statement 1: \\
    Let \(x\) be an element of \((A - B) \cap (C - B)\). \\
    Since \(x\) is in the intersection, it must be in both \(A - B\) and \(C - B\). Thus, \(x\) belongs to \(A\) and \(C\) but not to \(B\). Consequently, \(x\) is an element of \(A \cap C\) and not in \(B\), leading to \(x\) being in \((A \cap C) - B\). This confirms \((A - B) \cap (C - B) \subseteq (A \cap C) - B\). \\ \\
    Next, for statement 2: \\
    Take \(x\) as an element of \((A \cap C) - B\). \\
    From the set difference and intersection definitions, \(x\) is in both \(A\) and \(C\), and not in \(B\). Hence, \(x\) lies in both \(A - B\) and \(C - B\). Therefore, by intersection properties, \(x\) is in \((A - B) \cap (C - B)\). \\
    This confirms \((A \cap C) - B \subseteq (A - B) \cap (C - B)\).
    Since both sub-statements are proven, it follows that \((A - B) \cap (C - B) = (A \cap C) - B\), thereby validating the original statement.
\end{proof}

\begin{question}
    {6.2.14}
    {
        Use an element argument to prove the statement.
        Assume that all sets are subsets of a universal set $U$.
        For all sets $A, B$, and $C$, if $A \subseteq B$ then $A \cup C \subseteq B \cup C$.
    }
\end{question}
\begin{proof}
    Assume sets \(A, B, C\) with \(A \subseteq B\). Consider an arbitrary element \(x\). Assume \(x \in A \cup C\). According to the definition of union, this implies \(x \in A\) or \(x \in C\). \\ \\
    \textbf{\textit{Case 1:}} \(x \in A\): \\
    Given \(A \subseteq B\), if \(x\) is in \(A\), it must also be in \(B\). Hence, by the nature of union, \(x\) is in \(B \cup C\). \\ \\
    \textbf{\textit{Case 2:}} \(x \in C\): \\
    Directly by the definition of union, if \(x\) is in \(C\), it is necessarily in \(B \cup C\).
    In both scenarios, \(x \in B \cup C\) is validated. Thus, it is concluded that \(A \cup C \subseteq B \cup C\).
\end{proof}

\begin{question}
    {6.2.32}
    {
        Use the element method for proving a set equals the empty set to prove the statement.
        Assume that all sets are subsets of a universal set $U$.
        For all sets $A, B$, and $C$, if $A \subseteq B$ and $B \cap C=\emptyset$ then $A \cap C=\emptyset$.
    }
\end{question}
\begin{proof}
    To prove the statement, assume the contrary. Consider sets \(A, B, C\) such that \(A \subseteq B\) and \(B \cap C = \emptyset\). The negation of the statement is \(A \cap C \neq \emptyset\), implying the existence of an element \(x\) such that \(x \in A\) and \(x \in C\). \\ \\
    Since \(x \in A\) and \(A \subseteq B\), it follows that \(x \in B\). However, if \(x \in C\) and \(B \cap C = \emptyset\), this leads to a contradiction because \(x\) cannot be in both \(B\) and \(C\) if their intersection is empty. \\ \\
    This contradiction disproves the negation, thereby proving the original statement: If \(A \subseteq B\) and \(B \cap C = \emptyset\), then \(A \cap C = \emptyset\).
\end{proof}

\begin{question}
    {6.2.39}
    {
        Prove the statement.
        For all integers $n \geq 1$, if $A_1, A_2, A_3, \ldots$ and $B$ are any sets, then
        \begin{align*}
            \bigcap_{i=1}^n (A_i - B) = \left(\bigcap_{i=1}^n A_i \right) - B \text{. }
        \end{align*}
    }
\end{question}
\begin{proof}
    Assume \(A_1, A_2, A_3, \ldots\) and \(B\) are arbitrary sets. The statement is proven by showing:
    \begin{enumerate}
        \item[1.] \(\left[\bigcap_{i = 1}^n (A_i - B)\right] \subseteq \left[\bigcap_{i = 1}^n A_i\right] - B\)
        \item[2.] \(\left[\bigcap_{i = 1}^n A_i\right] - B \subseteq \left[\bigcap_{i = 1}^n (A_i - B)\right]\)
    \end{enumerate}
    First, to prove statement 1: \\
    Let \(x\) be an element of \(\bigcap_{i = 1}^n (A_i - B)\). This means \(x\) is in each \(A_i - B\) for all \(i\), so \(x\) is in every \(A_i\) and not in \(B\). Therefore, \(x\) is in \(\bigcap_{i = 1}^n A_i\) and, by set difference, \(x\) is in \(\left(\bigcap_{i = 1}^n A_i\right) - B\). \\ \\
    Next, to prove statement 2: \\
    Consider \(x\) as an element of \((\bigcap_{i = 1}^n A_i) - B\). By set difference, \(x\) is in \(\bigcap_{i = 1}^n A_i\) and not in \(B\). Hence, \(x\) is in each \(A_i\) and not in \(B\), implying \(x\) is in every \(A_i - B\), and therefore in \(\bigcap_{i = 1}^n (A_i - B)\). \\ \\
    As both sub-statements are confirmed, it follows that \(\bigcap_{i = 1}^n (A_i - B) = \left(\bigcap_{i = 1}^n A_i\right) - B\), thus proving the original statement.
\end{proof}