\documentclass{article}
\usepackage[utf8]{inputenc}
\usepackage{amsmath}
\usepackage{amssymb}
\usepackage{xcolor} % for color definitions
\usepackage{mdframed} % for framing and shading the problems
\usepackage{lipsum} % for generating text, remove in your actual document
\usepackage{geometry} % for page layout
\usepackage{titling} % for title page layout
\usepackage{listings} % for code listings
\DeclareMathOperator{\lcm}{lcm} % for lcm operator

% Set the page size and margins
\geometry{letterpaper, portrait, margin=1in}

% Define a new environment for the problems that takes one argument for the problem number
\newenvironment{problem}[1]{
    \begin{mdframed}[backgroundcolor=gray!20, skipabove=\baselineskip, skipbelow=\baselineskip, nobreak=true, innerleftmargin=10pt, innerrightmargin=10pt, innertopmargin=10pt, innerbottommargin=10pt]
    \textbf{Problem #1.}
}{
    \end{mdframed}
}

% Define a new environment for the proofs
\newenvironment{proof}{
    \begin{mdframed}[nobreak=false, innerleftmargin=10pt, innerrightmargin=10pt, innertopmargin=10pt, innerbottommargin=10pt]
    \textbf{Proof.}
}{
    \hfill $\square$
    \end{mdframed}
}

% Remove section numbering
\makeatletter
\renewcommand{\@seccntformat}[1]{}
\makeatother

% Remove table of contents numbering
\renewcommand{\thesection}{}

% Title page info
\title{Capstone: Discrete Math \\ \large Homework 9}
\author{Ojas Chaturvedi}
\date{Due: November 17, 2023}

% ------------------------------------------------------------------------------
\begin{document}

% Title page
\begin{titlingpage}
    \maketitle
    \tableofcontents
\end{titlingpage}

% ------------------------------------------------------------------------------

\section{Homework: 5.3}
    \begin{problem}{12}
        Prove the statement by mathematical induction:
        \begin{align*}
            \text{For any integer } n \geq 0, 7^n-2^n \text{is divisible by } 5.
        \end{align*}
    \end{problem}
    \begin{proof}
        Let the property $P(n)$ be the statement:
        \begin{align*}
            7^n-2^n \text{is divisible by } 5
        \end{align*}
        First, we must prove that $P(0)$ is true (basis step).
        \begin{align*}
            7^0-2^0 \text{ is divisible by } 5
        \end{align*}
        \begin{align*}
            7^0-2^0 &= 1-1 \\
            &= 0 \\
            &= 5(0)
        \end{align*}
        Thus, $P(0)$ is true. \\
        Now, suppose that $P(k)$ is true for some integer $k \geq 0$ (inductive hypothesis). That is,
        \begin{align*}
            7^k-2^k \text{ is divisible by } 5
        \end{align*}
        By the definition of divisibility, for some integer $r$,
        \begin{align*}
            7^k-2^k &= 5r
        \end{align*}
        We must show that $P(k+1)$ is true (inductive step). That is,
        \begin{align*}
            7^{k+1}-2^{k+1} \text{ is divisible by } 5
        \end{align*}
        The left-hand side of $P(k+1)$ is:
        \begin{align*}
            7^{k+1}-2^{k+1} &= 7(7^k)-2(2^k) \\
            &= 5\cdot7^k + 2\cdot7^k - 2\cdot2^k \\
            &= 5\cdot7^k + 2(7^k - 2^k) \\
            &= 5\cdot7^k + 2(5r) \\
            &= 5(7^k + 2r)
        \end{align*}
        $7^k + 2r$ is an integer since it is a sum of the product of integers, so $7^{k+1}-2^{k+1}$ can be written as $5m$ for some integer $m = (7^k + 2r)$. \\
        By the definition of divisibility, $7^{k+1}-2^{k+1}$ is divisible by $5$, and thus, $P(k+1)$ is true. \\
        Since we have proved both the basis step and the inductive step, we conclude that the statement is true using mathematical induction. 
    \end{proof}

    \begin{problem}{18}
        Prove the statement by mathematical induction:
        \begin{align*}
            5^n + 9 < 6^n, \text{ for all integers } n \geq 2.
        \end{align*}
    \end{problem}
    \begin{proof}
        Let the property $P(n)$ be the inequality:
        \begin{align*}
            5^n + 9 < 6^n
        \end{align*}
        First, we must prove that $P(0)$ is true (basis step).
        \begin{align*}
            5^2 + 9 &< 6^2 \\
            25 + 9 &< 36 \\
            34 &< 36
        \end{align*}
        Thus, $P(0)$ is true. \\
        Now, suppose that $P(k)$ is true for some integer $k \geq 2$ (inductive hypothesis). That is,
        \begin{align*}
            5^k + 9 < 6^k
        \end{align*}
        We must show that $P(k+1)$ is true (inductive step). That is,
        \begin{align*}
            5^{k+1} + 9 &< 6^{k+1}
        \end{align*}
        The left-hand side of $P(k+1)$ is:
        \begin{align*}
            5^{k+1} + 9 &= 5(5^k) + 9 \\
            &= 5 \cdot (5^k + 9 - 9) + 9 \\
            &= 5 \cdot ((5^k + 9) - 9) + 9 \\
            &< 5 \cdot (6^k-9) + 9 \\
            &< 5 \cdot 6^k - 45 + 9 \\
            &< 5 \cdot 6^k - 36 \\
            &< 5 \cdot 6^k \\
            &< 6 \cdot 6^k \\
            &< 6^{k+1}
        \end{align*}
        Thus, the left-hand side of $P(k+1)$ is less than the right-hand side of $P(k+1)$, and $P(k+1)$ is true. \\
        Since we have proved both the basis step and the inductive step, we conclude that the statement is true using mathematical induction.
    \end{proof}

    \begin{problem}{22}
        Prove the statement by mathematical induction:
        \begin{align*}
            1 + nx \leq {(1 + x)}^n, \text{ for all real numbers } x > -1 \text{ and integers } n \geq 2.
        \end{align*}
    \end{problem}
    \begin{proof}
        Let the property $P(n)$ be the inequality:
        \begin{align*}
            1 + nx \leq {(1 + x)}^n
        \end{align*}
        First, we must prove that $P(2)$ is true (basis step).
        \begin{align*}
            1 + 2x &= 1 + 2x + 0\\
            &\leq 1 + 2x + x^2 \\
            &= {(1 + x)}^2 \\
            &= {(1 + x)}^n
        \end{align*}
        Thus, $P(0)$ is true. \\
        Now, suppose that $P(k)$ is true for some integer $k \geq 2$ (inductive hypothesis). That is,
        \begin{align*}
            1 + kx \leq {(1 + x)}^k
        \end{align*}
        We must show that $P(k+1)$ is true (inductive step). That is,
        \begin{align*}
            1 + {(k+1)}x \leq {(1 + x)}^{(k+1)}
        \end{align*}
        The right-hand side of $P(k+1)$ is:
        \begin{align*}
            (1+x)^{k+1}&=(1+x)^{k}(1+x) \\
            &\geq(1+k x)(1+x) \\
            &=(1)(1)+(k x)(1)+(1)(x)+(k x)(x) \\
            &=1+k x+x+k x^{2} \\
            &=1+(k+1) x+k x^{2} \\
            &\geq 1+(k+1) x+0 \\
            &=1+(k+1) x
        \end{align*}
        Thus, the right-hand side of $P(k+1)$ is greater than or equal to the left-hand side of $P(k+1)$, and $P(k+1)$ is true. \\
        Since we have proved both the basis step and the inductive step, we conclude that the statement is true using mathematical induction.
    \end{proof}

\newpage \section{Homework: 5.4}
    \begin{problem}{2}
        Suppose $b_1, b_2, b_3, \ldots$ is a sequence defined as follows:
        \begin{align*}
            b_1 &= 4 \\
            b_2 &= 12 \\
            b_k &= b_{k-2} + b_{k-1} & \text{ for all integers } k \geq 3.
        \end{align*}
        Prove that $b_n$ is divisible by $4$ for all integers $n \geq 1$.
    \end{problem}
    \begin{proof}
        We need to prove that $b_n$ is divisible by 4 for all integers $n \geq 1$ in the sequence defined by
        \begin{align*}
            b_1 &= 4 \\
            b_2 &= 12 \\
            b_k &= b_{k-2} + b_{k-1} & \text{ for all integers } k \geq 3.
        \end{align*}
        \textbf{Basis Step:}
        For $n = 1$, $b_1 = 4$, which is divisible by 4. \\
        For $n = 2$, $b_2 = 12$, which is also divisible by 4. \\ \\
        \textbf{Inductive Step:}
        Assume the statement is true for some integers $k$ and $k-1$, i.e., both $b_k$ and $b_{k-1}$ are divisible by 4. We must show it is true for $k+1$. \\
        From the recursive definition,
        \begin{align*}
            b_{k+1} = b_k + b_{k-1}
        \end{align*}
        If $b_k = 4m$ and $b_{k-1} = 4n$ for some integers $m$ and $n$, then
        \begin{align*}
            b_{k+1} = 4m + 4n = 4(m + n)
        \end{align*}
        which is divisible by 4. \\
        Hence, by mathematical induction, $b_n$ is divisible by 4 for all integers $n \geq 1$.
    \end{proof}

    \begin{problem}{3}
        Suppose that $c_0, c_1, c_2, \ldots$ is a sequence defined as follows:
        \begin{align*}
            c_0 &= 2 \\
            c_1 &= 2 \\
            c_2 &= 6 \\
            c_k &= 3c_{k-3} & \text{ for all integers } k \geq 3.
        \end{align*}
        Prove that $c_n$ is even for all integers $n \geq 0$.
    \end{problem}
    \begin{proof}
        We need to prove that $c_n$ is even for all integers $n \geq 0$ in the sequence defined by
        \begin{align*}
            c_0 &= 2 \\
            c_1 &= 2 \\
            c_2 &= 6 \\
            c_k &= 3c_{k-3} & \text{ for all integers } k \geq 3.
        \end{align*}
        \textbf{Basis Step:}
        For $n = 0$, $c_0 = 2$, which is even. \\
        For $n = 1$, $c_1 = 2$, which is even. \\
        For $n = 2$, $c_2 = 6$, which is even. \\
        \textbf{Inductive Step:}
        Assume the statement is true for $n = k-3$, i.e., $c_{k-3}$ is even. We must show it is true for $k$. \\
        From the recursive definition,
        \begin{align*}
            c_k = 3c_{k-3}
        \end{align*}
        If $c_{k-3} = 2m$ for some integer $m$, then
        \begin{align*}
            c_k = 3 \cdot 2m = 2 \cdot (3m)
        \end{align*}
        which is divisible by 2, hence even. \\
        Therefore, by mathematical induction, $c_n$ is even for all integers $n \geq 0$.
    \end{proof}

    \begin{problem}{7}
        Suppose that $g_1, g_2, g_3, \ldots$ is a sequence defined as follows:
        \begin{align*}
            g_1 = 3 \\
            g_2 = 5 \\
            g_k = 3g_{k-1} - 2g_{k-2} & \text{ for all integers } k \geq 3.
        \end{align*}
        Prove that $g_n = 2^n + 1$ for all integers $n \geq 1$.
    \end{problem}
    \begin{proof}
        We need to prove that $g_n = 2^n + 1$ for all integers $n \geq 1$ in the sequence defined by
        \begin{align*}
            g_1 = 3 \\
            g_2 = 5 \\
            g_k = 3g_{k-1} - 2g_{k-2} & \text{ for all integers } k \geq 3.
        \end{align*}
        \textbf{Basis Step:}
        For $n = 1$, $g_1 = 3$ equals $2^1 + 1 = 3$. \\
        For $n = 2$, $g_2 = 5$ equals $2^2 + 1 = 5$. \\
        \textbf{Inductive Step:}
        Assume the statement is true for $n = k-1$ and $n = k-2$, i.e., $g_{k-1} = 2^{k-1} + 1$ and $g_{k-2} = 2^{k-2} + 1$. We must show it is true for $k$. \\
        From the recursive definition,
        \begin{align*}
            g_k = 3g_{k-1} - 2g_{k-2}
        \end{align*}
        Substituting the inductive hypothesis,
        \begin{align*}
            g_k = 3(2^{k-1} + 1) - 2(2^{k-2} + 1)
        \end{align*}
        Expanding and simplifying,
        \begin{align*}
            g_k = 3 \cdot 2^{k-1} + 3 - 2^{k-1} - 2 \\
            g_k = 2 \cdot 2^{k-1} + 1 \\
            g_k = 2^k + 1
        \end{align*}
        Thus, $g_k = 2^k + 1$.
        Therefore, by mathematical induction, $g_n = 2^n + 1$ for all integers $n \geq 1$.
    \end{proof}

    \begin{problem}{18}
        Compute $9^0, 9^1, 9^2, 9^3, 9^4$ and $9^5$. Make a conjecture about the units digit of $9^n$ where $n$ is a positive integer. Use strong mathematical induction to prove your conjecture.
    \end{problem}
    \begin{proof}
        Compute the powers of 9:
        \begin{align*}
            9^0 &= 1 \\
            9^1 &= 9 \\
            9^2 &= 81 \\
            9^3 &= 729 \\
            9^4 &= 6561 \\
            9^5 &= 59049
        \end{align*}
        \textbf{Conjecture:}
        The units digit of $9^n$ alternates between 1 and 9 for positive integers $n$. \\ \\ \\
        \textbf{Proof by Strong Mathematical Induction:} \\ \\
        \textbf{Basis Step:}
            The conjecture holds for $n = 0$ and $n = 1$. \\
        \textbf{Inductive Step:}
            Assume the conjecture holds for all integers less than or equal to $k$ for some $k \geq 1$. We need to show it holds for $k+1$.
            \begin{enumerate}
                \item If $k+1$ is even, then $k$ is odd, and by the inductive hypothesis, the units digit of $9^k$ is 9. Thus, the units digit of $9^{k+1} = 9^k \times 9$ is 1.
                \item If $k+1$ is odd, then $k$ is even, and by the inductive hypothesis, the units digit of $9^k$ is 1. Thus, the units digit of $9^{k+1} = 9^k \times 9$ is 9.
            \end{enumerate}
        By strong mathematical induction, the conjecture is true for all integers $n \geq 0$.
    \end{proof}

\newpage \section{Homework: 5.5}
    \begin{problem}{2}
        Show that if the predicate is true before entry to the loop, then it is also true after exit from the loop: \\
        Loop:
        \begin{lstlisting}[language=Python]
            while (m >= 0 and m <= 100)
                m := m + 4
                n := n - 2
            end while
        \end{lstlisting}
        Predicate: $m+n$ is odd
    \end{problem}
    \begin{proof}
        We need to show that if the predicate ``$m + n$ is odd'' is true before entry to the loop, then it is also true after exit from the loop. \\
        The loop performs the following operations while $m$ is between 0 and 100 (inclusive):
        \begin{lstlisting}[language=Python]
        while (m >= 0 and m <= 100)
            m := m + 4
            n := n - 2
        end while
        \end{lstlisting}
        The changes to $m$ and $n$ are by even numbers (4 and 2, respectively). \\
        If $m + n$ is odd initially, then adding even numbers to both $m$ and $n$ results in an even change to $m + n$. Since adding an even number to an odd number results in an odd number, $m + n$ remains odd after each iteration and after the loop exits. \\
        Therefore, if the predicate ``$m + n$ is odd'' is true before the loop, it remains true after the loop exits.
    \end{proof}

    \begin{problem}{4}
        Show that if the predicate is true before entry to the loop, then it is also true after exit from the loop: \\
        Loop:
        \begin{lstlisting}[language=Python]
            while (n >= 0 and n <= 100)
                n := n + 1
            end while
        \end{lstlisting}
        Predicate: $2^n < (n+2)! $
    \end{problem}
    \begin{proof}
        We need to show that if the predicate $2^n < (n+2)!$ is true before entry to the loop, then it is also true after exit from the loop. \\
        The loop performs the following operation while $n$ is between 0 and 100 (inclusive):
        \begin{lstlisting}[language=Python]
        while (n >= 0 and n <= 100)
            n := n + 1
        end while
        \end{lstlisting}
        Assume the predicate $2^n < (n+2)!$ is true for some $n$. We need to show it holds for $n+1$, i.e., $2^{n+1} < {(n+3)} ! $ \\
        Since $2^n < (n+2)!$, multiplying both sides by 2 gives $2 \cdot 2^n < 2 \cdot (n+2)!$. Also, $2 < (n+3)$ for all $n \geq 0$, hence $2 \cdot (n+2)! < (n+3) \cdot (n+2)! = (n+3)!$. \\
        Thus, $2^{n+1} < (n+3)!$, proving the predicate for $n+1$. \\
        Therefore, if the predicate is true before the loop, it remains true after the loop exits.
    \end{proof}

\newpage \section{Homework: 5.6}
    \begin{problem}{2}
        Find the first four terms of the recursively defined sequence:
        \begin{align*}
            b_k = b_{k-1} + 3k & \text{, for all integers } k \geq 2 \\
            b_1 = 1
        \end{align*}
    \end{problem}
    \begin{proof}
        \begin{align*}
            b_1 &= 1 \\
            b_2 &= b_{1} + 3 \times 2 = 1 + 6 = 7 \\
            b_3 &= b_{2} + 3 \times 3 = 7 + 9 = 16 \\
            b_4 &= b_{3} + 3 \times 4 = 16 + 12 = 28
        \end{align*}
    \end{proof}
    
    \begin{problem}{6}
        Find the first four terms of the recursively defined sequence:
        \begin{align*}
            t_k = t_{k-1} + 2t_{k-2} & \text{, for all integers } k \geq 2 \\
            t_0 = -1 \\
            t_1 = 2
        \end{align*}
    \end{problem}
    \begin{proof}
        \begin{align*}
            t_0 &= -1 \\
            t_1 &= 2 \\
            t_2 &= t_{1} + 2t_{0} = 2 + 2(-1) = 2 - 2 = 0 \\
            t_3 &= t_{2} + 2t_{1} = 0 + 2 \times 2 = 0 + 4 = 4
        \end{align*}
    \end{proof}

    \begin{problem}{8}
        Find the first four terms of the recursively defined sequence:
        \begin{align*}
            v_k = v_{k-1} + v_{k-2} + 1 & \text{, for all integers } k \geq 3 \\
            v_1 = 1 \\
            v_2 = 3
        \end{align*}
    \end{problem}
    \begin{proof}
        \begin{align*}
            v_1 &= 1 \\
            v_2 &= 3 \\
            v_3 &= v_{2} + v_{1} + 1 = 3 + 1 + 1 = 5 \\
            v_4 &= v_{3} + v_{2} + 1 = 5 + 3 + 1 = 9
        \end{align*}
    \end{proof}

    \begin{problem}{12}
        Let $s_0, s_1, s_2, \ldots$ be defined by the formula $s_n = \frac{(-1)^n}{n!}$ for all integers $n \geq 0$. Show that this sequence satisfies the recurrence relation
        \begin{align*}
            s_k = \frac{-s_{k-1}}{k}
        \end{align*}
    \end{problem}
    \begin{proof}
        Given that for all integers $n \geq 0$, we need to show that it satisfies $s_k = \frac{-s_{k-1}}{k}$. \\
        Observe,
        \begin{align*}
            s_k &= \frac{(-1)^k}{k!} \\
            &= \frac{(-1)^k}{k \cdot (k-1)!} \\
            &= \frac{-(-1)^{k-1}}{k \cdot (k-1)!} \\
            &= \frac{-1}{k} \cdot \frac{{(-1)}^{k-1}}{(k-1)!} \\
            &= \frac{-s_{k-1}}{k}
        \end{align*}
        Hence, the sequence $s_n$ satisfies the recurrence relation $s_k = \frac{-s_{k-1}}{k}$.
    \end{proof}

    \begin{problem}{14}
        Let $d_0, d_1, d_2, \ldots$ be defined by the formula $d_n = 3^n - 2^n$ for all integers $n \geq 0$. Show that this sequence satisfies the recurrence relation
        \begin{align*}
            d_k = 5d_{k-1} - 6d_{k-2}
        \end{align*}
    \end{problem}
    \begin{proof}
        Given that $d_n = 3^n - 2^n$ for all integers $n \geq 0$, we need to show that it satisfies $d_k = 5d_{k-1} - 6d_{k-2}$. \\
        We have:
        \begin{align*}
            d_k &= 3^k - 2^k \\
            d_{k-1} &= 3^{k-1} - 2^{k-1} \\
            d_{k-2} &= 3^{k-2} - 2^{k-2}
        \end{align*}
        Now, calculate $5d_{k-1} - 6d_{k-2}$:
        \begin{align*}
            5d_{k-1} - 6d_{k-2} &= 5(3^{k-1} - 2^{k-1}) - 6(3^{k-2} - 2^{k-2}) \\
            &= 5 \cdot 3^{k-1} - 5 \cdot 2^{k-1} - 6 \cdot 3^{k-2} + 6 \cdot 2^{k-2} \\
            &= 3 \cdot 3^{k-2} \cdot 5 - 2 \cdot 2^{k-2} \cdot 5 - 3^{k-2} \cdot 6 + 2^{k-2} \cdot 6 \\
            &= 3^{k-2} (3 \cdot 5 - 6) + 2^{k-2} (6 - 2 \cdot 5) \\
            &= 3^{k-2} \cdot 9 + 2^{k-2} \cdot (-4) \\
            &= 3^k - 2^k
        \end{align*}
        Thus, $5d_{k-1} - 6d_{k-2} = 3^k - 2^k$, which is the formula for $d_k$. \\
        Therefore, the sequence $d_n$ satisfies the recurrence relation $d_k = 5d_{k-1} - 6d_{k-2}$.
    \end{proof}
    

    \begin{problem}{28}
        Prove that $F_{k+1}^2 - F_k^2 - F_{k-1}^2 = 2F_k F_{k-1}$, for all integers $k \geq 1$.
    \end{problem}
    \begin{proof}
        Consider the Fibonacci sequence defined by $F_n = F_{n-1} + F_{n-2}$ with $F_0 = 0$ and $F_1 = 1$. \\
        We need to prove that $F_{k+1}^2 - F_k^2 - F_{k-1}^2 = 2F_k F_{k-1}$ for all integers $k \geq 1$. \\
        The left-hand side of the equation is $F_{k+1}^2 - F_k^2 - F_{k-1}^2$. \\
        Using the Fibonacci recurrence relation, $F_{k+1} = F_k + F_{k-1}$, we have:
        \begin{align*}
        F_{k+1}^2 &= (F_k + F_{k-1})^2 \\
                  &= F_k^2 + 2F_k F_{k-1} + F_{k-1}^2
        \end{align*}
        Substituting this into the left-hand side of the equation gives:
        \begin{align*}
        F_{k+1}^2 - F_k^2 - F_{k-1}^2 &= F_k^2 + 2F_k F_{k-1} + F_{k-1}^2 - F_k^2 - F_{k-1}^2 \\
        &= 2F_k F_{k-1}
        \end{align*}
        which is equal to the right-hand side of the given equation. \\
        Therefore, $F_{k+1}^2 - F_k^2 - F_{k-1}^2 = 2F_k F_{k-1}$ is proven for all integers $k \geq 1$.
    \end{proof}

    \begin{problem}{44}
        The triangle inequality for absolute value states that for all real numbers $a$ and $b$, $|a + b| \leq |a| + |b|$. Use the recursive definition of summation, the triangle inequality, the definition of absolute value, and mathematical induction to prove that for all positive integers $n$, if $a_1, a_2, \ldots, a_n$ are real numbers, then
        \begin{align*}
            \left|\sum_{i=1}^{n}a_i \right| \leq \sum_{i=1}^{n}|a_i|
        \end{align*}
    \end{problem}
    \begin{proof}
        We need to prove by induction that for all positive integers $n$, if $a_1, a_2, \ldots, a_n$ are real numbers, then
        \begin{align*}
            \left|\sum_{i=1}^{n}a_i \right| \leq \sum_{i=1}^{n}|a_i|
        \end{align*}
        \textbf{Basis Step:}
        For $n = 1$, we have $|a_1| \leq |a_1|$, which is trivially true. \\
        \textbf{Inductive Step:}
        Assume the statement is true for some integer $k \geq 1$, i.e., 
        \[ \left|\sum_{i=1}^{k}a_i \right| \leq \sum_{i=1}^{k}|a_i| \]
        We must show it is true for $k + 1$. \\
        Consider
        \[ \sum_{i=1}^{k+1}a_i = \sum_{i=1}^{k}a_i + a_{k+1} \]
        By the triangle inequality,
        \[ \left|\sum_{i=1}^{k+1}a_i \right| = \left|\sum_{i=1}^{k}a_i + a_{k+1}\right| \leq \left|\sum_{i=1}^{k}a_i\right| + |a_{k+1}| \]
        Using the inductive hypothesis,
        \[ \left|\sum_{i=1}^{k}a_i\right| \leq \sum_{i=1}^{k}|a_i| \]
        Thus,
        \[ \left|\sum_{i=1}^{k}a_i\right| + |a_{k+1}| \leq \sum_{i=1}^{k}|a_i| + |a_{k+1}| = \sum_{i=1}^{k+1}|a_i| \]
        Hence, by mathematical induction, the statement holds for all positive integers $n$.
    \end{proof}


% ------------------------------------------------------------------------------
\end{document}
% ------------------------------------------------------------------------------