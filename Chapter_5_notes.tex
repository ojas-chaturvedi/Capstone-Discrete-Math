\documentclass{article}
\usepackage[utf8]{inputenc}
\usepackage{amsmath}
\usepackage{amssymb}
\usepackage{xcolor} % for color definitions
\usepackage{mdframed} % for framing and shading the problems
\usepackage{lipsum} % for generating text, remove in your actual document
\usepackage{geometry} % for page layout
\usepackage{titling} % for title page layout

% Set the page size and margins
\geometry{letterpaper, portrait, margin=1in}

% Define a new environment for the problems that takes one argument for the problem number
\newenvironment{problem}[1]{
    \begin{mdframed}[backgroundcolor=gray!20, skipabove=\baselineskip, skipbelow=\baselineskip, nobreak=true, innerleftmargin=10pt, innerrightmargin=10pt, innertopmargin=10pt, innerbottommargin=10pt]
    \textbf{Problem #1.}
}{
    \end{mdframed}
}

% Define a new environment for the proofs
\newenvironment{proof}{
    \begin{mdframed}[nobreak=true, innerleftmargin=10pt, innerrightmargin=10pt, innertopmargin=10pt, innerbottommargin=10pt]
    \textbf{Proof.}
}{
    \hfill $\square$
    \end{mdframed}
}

% Remove section numbering
\makeatletter
\renewcommand{\@seccntformat}[1]{}
\makeatother

% Remove table of contents numbering
\renewcommand{\thesection}{}
\renewcommand{\thesubsection}{} % remove subsection numbering

% Title page info
\title{Chapter 5 \\ \large Sequences, Mathematical Induction, and Recursion}
\author{Ojas Chaturvedi}
\date{}

% ------------------------------------------------------------------------------
\begin{document}

% Title page
\begin{titlingpage}
    \maketitle
    \tableofcontents
\end{titlingpage}

% ------------------------------------------------------------------------------
\section{5.1: Sequences}
    \subsection{Notes}
        \begin{itemize}
            \item Sequence: a function whose domain is either all the integers between two given integers, or all the integers greater than or equal to a given integer.
            \begin{itemize}
                \item Know subscript/index, initial and final term, infinite sequence, general/explicit formula
            \end{itemize}
            \item Summation Notation:
                \begin{align*}
                    \sum_{k=m}^{n} a_k = a_m + a_{m+1} + a_{m+2} + \cdots + a_n
                \end{align*}
                where k is the index, m is the lower limit, and n is the upper limit.
            \item When the upper limit of a summation is a variable, an ellipsis is used to write the summation in expanded form. Expand the summation notation to first 3 or so, then put ellipsis and then variable form.
            \item Product Notation:
                \begin{align*}
                    \prod_{k=m}^{n} a_k = a_m \cdot a_{m+1} \cdot a_{m+2} \cdots a_n
                \end{align*}
            \item Properties of Summations and Products (aka Theorem 5.1.1)
                \begin{align}
                    \sum_{k=m}^{n} a_k + \sum_{k=m}^{n} b_k &= \sum_{k=m}^{n} (a_k + b_k) \\
                    c \cdot \sum_{k=m}^{n} a_k &= \sum_{k=m}^{n} (c \cdot a_k) \\
                    (\prod_{k=m}^{n} a_k) \cdot (\prod_{k=m}^{n} b_k) &= \prod_{k=m}^{n} (a_k \cdot b_k)
                \end{align}
            \item When replacing a new variable into a summation or product, make sure to change the index variable to the new variable and the numbers by putting them into the equation of the new variable.
            \item Factorial: the quantity $\mathbf{n!}$ is defined to be the product of all the integers from 1 to n:
                \begin{align*}
                    n! = n \cdot (n-1) \cdots 3 \cdot 2 \cdot 1
                \end{align*}
                and
                \begin{align*}
                    0! = 1
                \end{align*}
                Recursive definition:
                \begin{align*}
                    n! = \begin{cases}
                        1 & \text{if } n = 0 \\
                        n \cdot (n-1)! & \text{if } n > 0
                    \end{cases}
                \end{align*}
            \item $n$ choose $r$: the number of subsets (therefore an integer) of size $r$ that can be chosen from a set of $n$ elements.
                \begin{align*}
                    n \choose r &= \frac{n!}{r!(n-r)!}
                \end{align*}
                for all integers $n$ and $r$ with $0 \leq r \leq n$.
        \end{itemize}

\newpage \section{5.2: Mathematical Induction I}
    \subsection{Notes}
        \begin{itemize}
            \item Principles of Mathematical Induction: Let $P(n)$ be a property that is defined for integers $n$, and let $a$ be a fixed integer. Suppose the following two statements are true:
                \begin{enumerate}
                    \item Basis Step: Show that $P(a)$ is true.
                    \item Inductive Step: For all integers $k \geq a$, if $P(k)$ is true, then $P(k+1)$ is true.
                        \begin{itemize}
                            \item To perform this step:
                                \begin{enumerate}
                                    \item Suppose that $P(k)$ is true for an arbitrary integer $k \geq a$, which is called the inductive hypothesis.
                                    \item Show that $P(k+1)$ is true.
                                \end{enumerate}
                            \item Remember that you need to prove each side of the equation separately. Otherwise, the proof is invalid.
                        \end{itemize}
                    \item Conclusion: Then $P(n)$ is true for all integers $n \geq a$.
                \end{enumerate}
            \item Steps of Proof by Mathematical Induction:
                \begin{enumerate}
                    \item State the theorem to be proved.
                        \begin{itemize}
                            \item Let the property $P(n)$ be the equation: {problem goes here}
                        \end{itemize}
                    \item Prove the basis step.
                        \begin{itemize}
                            \item Show that $P(a)$ is true.
                        \end{itemize}
                    \item State the inductive hypothesis.
                        \begin{itemize}
                            \item Show that for all integers $k \geq 1$, if $P(k)$ is true then $P(k+1)$ is also true:
                        \end{itemize}
                    \item Prove the inductive step.
                    \item State the conclusion.
                        \begin{itemize}
                            \item Therefore the equation $P(k+1)$ is true [\textit{as was to be shown}]. [\textit{Since we have proved both the basis step and the inductive step, the conclusion follows by the principle of mathematical induction. Therefore the equation $P(n)$ is true for all integers $n \geq 1$.}]
                        \end{itemize}
                \end{enumerate}
            \item Sum of the first $n$ integers is
                \begin{align*}
                    1 + 2 + 3 + \cdots + n = \frac{n(n+1)}{2}
                \end{align*}
            \item Geometric sum of the first $n$ integers is
                \begin{align*}
                    \sum_{i=0}^{n} r^i = \frac{r^{n+1}-1}{r-1}
                \end{align*}
        \end{itemize}

\newpage \section{5.3: Mathematical Induction II}
    \subsection{Different types of problems}
        \begin{problem}{Type: Divisibility Property}
            For all integers $n \geq 0$, $2^{2n} - 1$ is divisible by $3$.
        \end{problem}
        \begin{proof}
            Let the property $P(n)$ be the sentence:
            \begin{align*}
                2^{2n} - 1 \text{ is divisible by } 3
            \end{align*}
            First, we must prove that $P(0)$ is true (basis step).
            \begin{align*}
                2^{2 \cdot 0} - 1 \text{ is divisible by } 3
            \end{align*}
            \begin{align*}
                2^{2(0)} - 1 &= 2^0 - 1 \\
                &= 1 - 1 \\
                &= 0 \\
                &= 3 \cdot 0 \\
            \end{align*}
            Thus, $P(0)$ is true. \\
            Now, suppose that $P(k)$ is true for some integer $k \geq 0$ (inductive hypothesis). That is,
            \begin{align*}
                2^{2k} - 1 \text{ is divisible by } 3
            \end{align*}
            By the definition of divisibility, for some integer $r$,
            \begin{align*}
                2^{2k} - 1 = 3r
            \end{align*}
            We must show that $P(k+1)$ is true (inductive step). That is,
            \begin{align*}
                2^{2(k+1)} - 1 \text{ is divisible by } 3
            \end{align*}
            The left-hand side of $P(k+1)$ is:
            \begin{align*}
                2^{2(k+1)} - 1 &= 2^{2k+2} - 1 \\
                &= 2^{2k} \cdot 2^2 - 1 \\
                & \quad \text{by the product rule for exponents} \\
                &= 4 \cdot 2^{2k} - 1 \\
                &= 3 \cdot 2^{2k} + 2^{2k} - 1 \\
                &= 3 \cdot 2^{2k} + 3r \\
                & \quad \text{by substituting the inductive hypothesis} \\
                &= 3(2^{2k} + r)
            \end{align*}
            $2^{2k}+r$ is an integer since it is the sum of products of integers, so $2^{2(k+1)} - 1$ can be written as $6m$ for some integer $m = (2^{2k}+r)$. \\
            By the definition of divisibility, $2^{2(k+1)} - 1$ is divisible by $3$, and thus, $P(k+1)$ is true. \\
            Since we have proved both the basis step and the inductive step, we conclude that the statement is true using mathematical induction.
        \end{proof}

        \begin{problem}{Type: Inequality}
            For all integers $n \geq 3$, $2n+1 < 2^n$.
        \end{problem}
        \begin{proof}
            Let the property $P(n)$ be the inequality:
            \begin{align*}
                2n+1 < 2^n
            \end{align*}
            First, we must prove that $P(3)$ is true (basis step).
            \begin{align*}
                2(3)+1 < 2^3 \\
                7 < 8
            \end{align*}
            Thus, $P(3)$ is true. \\
            Now, suppose that $P(k)$ is true for some integer $k \geq 3$ (inductive hypothesis). That is,
            \begin{align*}
                2k+1 < 2^k
            \end{align*}
            We must show that $P(k+1)$ is true (inductive step). That is,
            \begin{align*}
                2(k+1)+1 < 2^{k+1} \\
                2k+3 < 2^{k+1}
            \end{align*}
            The left-hand side of $P(k+1)$ is:
            \begin{align*}
                2k+3 &= 2k+1+2 \\
                &< 2^k+2 \\
                & \quad \text{by substitution of the inductive hypothesis} \\
                &< 2^k+2^k \\
                & \quad \text{because $2 < 2^k$ for all integers $k \geq 2$} \\
                &< 2 \cdot 2^k \\
                &< 2^{k+1} \\
                & \quad \text{by the product rule for exponents}
            \end{align*}
            Thus, the left-hand side of $P(k+1)$ is less than the right-hand side of $P(k+1)$, and $P(k+1)$ is true. \\
            Since we have proved both the basis step and the inductive step, we conclude that the statement is true using mathematical induction.
        \end{proof}

        \begin{problem}{Type: Property of a Sequence}
            Define a sequence $a_1, a_2, a_3, \ldots$ as follows:
            \begin{align*}
                a_1 &= 2 \\
                a_k &= 5a_{k-1} & \text{ for all integers } k \geq 2
            \end{align*}
            \begin{enumerate}
                \item[a.] Write the first four terms of the sequence.
                \item[b.] It is claimed that for each integer $n \geq 0$, the $n$th term of the sequence has the same value as that given by the formula $2 \cdot 5^{n-1}$. In other words, the claim is that the terms of the sequence satisfy the equation $a_n = 2 \cdot 5^{n-1}$. Prove that this is true.
            \end{enumerate}
        \end{problem}
        \begin{proof}
            \begin{enumerate}
                \item[a.]
                    $a_1 = 2$ \\
                    $a_2 = 5a_{2-1} = 5a_1 = 5 \cdot 2 = 10$ \\
                    $a_3 = 5a_{3-1} = 5a_2 = 5 \cdot 10 = 50$ \\
                    $a_4 = 5a_{4-1} = 5a_3 = 5 \cdot 50 = 250$
                \item[b.]
                    Let $a_1, a_2, a_3, \ldots$ be the sequence defined by specifying that $a_1 = 2$ and $a_k = 5a_{k-1}$ for all integers $k \geq 2$. Let the property $P(n)$ be the equation:
                    \begin{align*}
                        a_n = 2 \cdot 5^{n-1}
                    \end{align*}
                    First, we must prove that $P(1)$ is true (basis step).
                    \begin{align*}
                        a_1 &= 2 \cdot 5^{1-1} \\
                    \end{align*}
                    The left-hand side of $P(1)$ is
                    \begin{align*}
                        a_1 &= 2 \\
                        & \quad \text{by the definition of $a_1, a_2, a_3, \ldots$} \\
                    \end{align*}
                    The right-hand side of $P(1)$ is
                    \begin{align*}
                        2 \cdot 5^{1-1} &= 2 \cdot 5^0 \\
                        &= 2 \cdot 1 \\
                        &= 2
                    \end{align*}
                    Thus, the left-hand side of $P(1)$ is equal to the right-hand side of $P(1)$, and $P(1)$ is true. \\
                    Now, suppose that $P(k)$ is true for some integer $k \geq 1$ (inductive hypothesis). That is,
                    \begin{align*}
                        a_k = 2 \cdot 5^{k-1}
                    \end{align*}
                    We must show that $P(k+1)$ is true (inductive step). That is,
                    \begin{align*}
                        a_{k+1} &= 2 \cdot 5^{(k+1)-1} \\
                        a_{k+1} &= 2 \cdot 5^k
                    \end{align*}
                    The left-hand side of $P(k+1)$ is:
                    \begin{align*}
                        a_{k+1} &= 5a_k \\
                        &= 5(2 \cdot 5^{k-1}) \\
                        &= 2 \cdot 5^k
                    \end{align*}
                    Thus, the left-hand side of $P(k+1)$ is equal to the right-hand side of $P(k+1)$, and $P(k+1)$ is true. \\
                    Since we have proved both the basis step and the inductive step, we conclude that the statement is true using mathematical induction.
            \end{enumerate}
        \end{proof}
        

\newpage \section{Template for Mathematical Induction}
    Let the property $P(n)$ be the equation/sentence/inequality:
    \begin{align*}
        \text{\{problem goes here\}}
    \end{align*}
    First, we must prove that $P(\text{\{smallest possible number goes here\}})$ is true (basis step).
    \begin{align*}
        \text{Show left-hand side = right-hand side of the equation}
    \end{align*}
    Thus, $P(\text{\{smallest possible number goes here\}})$ is true. \\
    Now, suppose that $P(k)$ is true for some integer $k \geq \text{\{smallest possible number goes here\}}$ (inductive hypothesis). That is, 
    \begin{align*}
        \text{\{problem with k substituted goes here\}}
    \end{align*}
    We must show that $P(k+1)$ is true (inductive step). That is, 
    \begin{align*}
        \text{\{problem with (k+1) substituted goes here\}}
    \end{align*}
    The left-hand side of $P(k+1)$ is:
    \begin{align*}
        \text{\{work with reasoning goes here\}}
    \end{align*}
    The right-hand side of $P(k+1)$ is:
    \begin{align*}
        \text{\{work with reasoning goes here\}}
    \end{align*}
    Thus, the left-hand side of $P(k+1)$ is equal to the right-hand side of $P(k+1)$, and $P(k+1)$ is true. \\
    Since we have proved both the basis step and the inductive step, we conclude that the statement is true using mathematical induction.
\end{document}
% ------------------------------------------------------------------------------