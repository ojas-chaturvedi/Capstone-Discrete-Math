\documentclass{article}
\usepackage[utf8]{inputenc}
\usepackage{amsmath}
\usepackage{amssymb}
\usepackage{xcolor} % for color definitions
\usepackage{mdframed} % for framing and shading the problems
\usepackage{lipsum} % for generating text, remove in your actual document
\usepackage{geometry} % for page layout
\usepackage{titling} % for title page layout

% Set the page size and margins
\geometry{letterpaper, portrait, margin=1in}

% Define a new environment for the problems that takes one argument for the problem number
\newenvironment{problem}[1]{
    \begin{mdframed}[backgroundcolor=gray!20, skipabove=\baselineskip, skipbelow=\baselineskip, nobreak=true, innerleftmargin=10pt, innerrightmargin=10pt, innertopmargin=10pt, innerbottommargin=10pt]
    \textbf{Problem #1.}
}{
    \end{mdframed}
}

% Define a new environment for the proofs
\newenvironment{proof}{
    \begin{mdframed}[nobreak=true, innerleftmargin=10pt, innerrightmargin=10pt, innertopmargin=10pt, innerbottommargin=10pt]
    \textbf{Proof.}
}{
    \hfill $\square$
    \end{mdframed}
}

% Remove section numbering
\makeatletter
\renewcommand{\@seccntformat}[1]{}
\makeatother

% Remove table of contents numbering
\renewcommand{\thesection}{}

% Title page info
\title{Chapter 5 \\ \large Sequences, Mathematical Induction, and Recursion}
\author{Ojas Chaturvedi}
\date{}

% ------------------------------------------------------------------------------
\begin{document}

% Title page
\begin{titlingpage}
    \maketitle
    \tableofcontents
\end{titlingpage}

% ------------------------------------------------------------------------------
\section{5.1: Sequences}
    \subsection{Notes}
        \begin{itemize}
            \item Sequence: a function whose domain is either all the integers between two given integers, or all the integers greater than or equal to a given integer.
            \begin{itemize}
                \item Know subscript/index, initial and final term, infinite sequence, general/explicit formula
            \end{itemize}
            \item Summation Notation:
                \begin{align*}
                    \sum_{k=m}^{n} a_k = a_m + a_{m+1} + a_{m+2} + \cdots + a_n
                \end{align*}
                where k is the index, m is the lower limit, and n is the upper limit.
            \item When the upper limit of a summation is a variable, an ellipsis is used to write the summation in expanded form. Expand the summation notation to first 3 or so, then put ellipsis and then variable form.
            \item Product Notation:
                \begin{align*}
                    \prod_{k=m}^{n} a_k = a_m \cdot a_{m+1} \cdot a_{m+2} \cdots a_n
                \end{align*}
            \item Properties of Summations and Products (aka Theorem 5.1.1)
                \begin{align}
                    \sum_{k=m}^{n} a_k + \sum_{k=m}^{n} b_k &= \sum_{k=m}^{n} (a_k + b_k) \\
                    c \cdot \sum_{k=m}^{n} a_k &= \sum_{k=m}^{n} (c \cdot a_k) \\
                    (\prod_{k=m}^{n} a_k) \cdot (\prod_{k=m}^{n} b_k) &= \prod_{k=m}^{n} (a_k \cdot b_k)
                \end{align}
            \item When replacing a new variable into a summation or product, make sure to change the index variable to the new variable and the numbers by putting them into the equation of the new variable.
            \item Factorial: the quantity $\mathbf{n!}$ is defined to be the product of all the integers from 1 to n:
                \begin{align*}
                    n! = n \cdot (n-1) \cdots 3 \cdot 2 \cdot 1
                \end{align*}
                and
                \begin{align*}
                    0! = 1
                \end{align*}
                Recursive definition:
                \begin{align*}
                    n! = \begin{cases}
                        1 & \text{if } n = 0 \\
                        n \cdot (n-1)! & \text{if } n > 0
                    \end{cases}
                \end{align*}
            \item $n$ choose $r$: the number of subsets (therefore an integer) of size $r$ that can be chosen from a set of $n$ elements.
                \begin{align*}
                    n \choose r &= \frac{n!}{r!(n-r)!}
                \end{align*}
                for all integers $n$ and $r$ with $0 \leq r \leq n$.
        \end{itemize}
    \subsection{Things to Check}
        \begin{itemize}
            \item Recursive definition of summation pg 232
        \end{itemize}

\newpage \section{5.2: Mathematical Induction I}
    \subsection{Notes}
        \begin{itemize}
            \item Principles of Mathematical Induction: Let $P(n)$ be a property that is defined for integers $n$, and let $a$ be a fixed integer. Suppose the following two statements are true:
                \begin{enumerate}
                    \item Basis Step: Show that $P(a)$ is true.
                    \item Inductive Step: For all integers $k \geq a$, if $P(k)$ is true, then $P(k+1)$ is true.
                        \begin{itemize}
                            \item To perform this step:
                                \begin{enumerate}
                                    \item Suppose that $P(k)$ is true for an arbitrary integer $k \geq a$, which is called the inductive hypothesis.
                                    \item Show that $P(k+1)$ is true.
                                \end{enumerate}
                            \item Remember that you need to prove each side of the equation separately. Otherwise, the proof is invalid.
                        \end{itemize}
                    \item Conclusion: Then $P(n)$ is true for all integers $n \geq a$.
                \end{enumerate}
            \item Steps of Proof by Mathematical Induction:
                \begin{enumerate}
                    \item State the theorem to be proved.
                        \begin{itemize}
                            \item Let the property $P(n)$ be the equation: {problem goes here}
                        \end{itemize}
                    \item Prove the basis step.
                        \begin{itemize}
                            \item Show that $P(a)$ is true.
                        \end{itemize}
                    \item State the inductive hypothesis.
                        \begin{itemize}
                            \item Show that for all integers $k \geq 1$, if $P(k)$ is true then $P(k+1)$ is also true:
                        \end{itemize}
                    \item Prove the inductive step.
                    \item State the conclusion.
                        \begin{itemize}
                            \item Therefore the equation $P(k+1)$ is true [\textit{as was to be shown}]. [\textit{Since we have proved both the basis step and the inductive step, the conclusion follows by the principle of mathematical induction. Therefore the equation $P(n)$ is true for all integers $n \geq 1$.}]
                        \end{itemize}
                \end{enumerate}
            \item Sum of the first $n$ integers is
                \begin{align*}
                    1 + 2 + 3 + \cdots + n = \frac{n(n+1)}{2}
                \end{align*}
            \item Geometric sum of the first $n$ integers is
                \begin{align*}
                    \sum_{i=0}^{n} r^i = \frac{r^{n+1}-1}{r-1}
                \end{align*}
        \end{itemize}
    \subsection{Things to check}
        \begin{itemize}
            \item Steps of proof and wording (check examples if need be) pg 247-248
            \item Check other method for solution to 5.2.4.b pg 283
            \item Cents problem pg 245
        \end{itemize}

\newpage \section{5.3: Mathematical Induction II}
    \subsection{Notes}
        \begin{itemize}
            \item Enter here.
        \end{itemize}

    \subsection{Things to check}
        \begin{itemize}
            \item 
        \end{itemize}

\end{document}