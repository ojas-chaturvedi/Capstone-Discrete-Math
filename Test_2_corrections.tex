\documentclass{article}
\usepackage[utf8]{inputenc}
\usepackage{amsmath}
\usepackage{amssymb}
\usepackage{xcolor} % for color definitions
\usepackage{mdframed} % for framing and shading the problems
\usepackage{lipsum} % for generating text, remove in your actual document
\usepackage{geometry} % for page layout
\usepackage{titling} % for title page layout

% Set the page size and margins
\geometry{letterpaper, portrait, margin=1in}

% Define a new environment for the problems that takes one argument for the problem number
\newenvironment{problem}[1]{
    \begin{mdframed}[backgroundcolor=gray!20, skipabove=\baselineskip, skipbelow=\baselineskip, nobreak=true, innerleftmargin=10pt, innerrightmargin=10pt, innertopmargin=10pt, innerbottommargin=10pt]
    \textbf{Problem #1.}
}{
    \end{mdframed}
}

% Define a new environment for the proofs
\newenvironment{proof}{
    \begin{mdframed}[nobreak=true, innerleftmargin=10pt, innerrightmargin=10pt, innertopmargin=10pt, innerbottommargin=10pt]
    \textbf{Proof.}
}{
    \hfill $\square$
    \end{mdframed}
}

% Remove section numbering
\makeatletter
\renewcommand{\@seccntformat}[1]{}
\makeatother

% Remove table of contents numbering
\renewcommand{\thesection}{}

% Title page info
\title{Test 2 Corrections \\ \large Capstone: Discrete Math}
\author{Ojas Chaturvedi \\ \small Absent for Model UN Conference}
\date{November 17, 2023}

% ------------------------------------------------------------------------------
\begin{document}

% Title page
\begin{titlingpage}
    \maketitle
\end{titlingpage}

% ------------------------------------------------------------------------------

\begin{problem}{2}
    For this problem, prove the Reverse Triangle Inequality, which is as follows: for all real numbers $x$ and $y$,
    \begin{equation*}
        |x| - |y| \leq |x + y|
    \end{equation*}
    You may use the Triangle Inequality itself as well as the following lemmas from class freely in your proof, indicating them by name if you apply them:
    \begin{itemize}
        \item \textbf{(Lemma 1)} For all real numbers $r$, $- |r| \leq r \leq |r|$
        \item \textbf{(Lemma 2)} For all real numbers $r$, $|-r| = |r|$
    \end{itemize}
    Hint: $|x| = |x + y + (-y)|$
\end{problem}
\begin{proof}
    We need to prove that for all real numbers $x$ and $y$, the inequality $|x| - |y| \leq |x + y|$ is true. \\
    By the Triangle Inequality, for all real numbers $a$ and $b$,
    \begin{equation*}
        |a + b| \leq |a| + |b|
    \end{equation*}
    Applying this to $x + y$ and $-y$, we obtain:
    \begin{align*}
        |(x + y) + (-y)| &\leq |x + y| + |-y| \\
        |x + y - y| &\leq |x + y| + |y| \\
        |x| &\leq |x + y| + |y| 
    \end{align*}
    By Lemma 2, $|-y| = |y|$. Thus,
    \begin{align*}
        |x| &\leq |x + y| + |y| \\
        |x| - |y| &\leq |x + y|
    \end{align*}
    Therefore, we have shown that $|x| - |y| \leq |x + y|$ for all real numbers $x$ and $y$.
\end{proof}

\begin{problem}{3}
    Prove that for all real numbers $x$, $\lceil{x} \rceil = - \lfloor{-x} \rfloor$.
\end{problem}
\begin{proof}
    We need to prove that $\lceil{x} \rceil = - \lfloor{-x} \rfloor$ for all real numbers $x$. \\
    By definition of ceiling, $\lceil{x}\rceil$ is the smallest integer greater than or equal to $x$.
    By definition of floor, $\lfloor{x}\rfloor$ is the greatest integer less than or equal to $x$. \\ \\
    Let $n$ be an integer such that:
    \begin{align*}
        n \leq x &< n + 1 \\
        -n - 1 < -x &\leq -n
    \end{align*}
    By definition of ceiling, $\lceil{x}\rceil = n+1$. 
    By definition of floor, $\lfloor{-x}\rfloor = -n-1$. \\
    Observe,
    \begin{align*}
        \lceil{x}\rceil &= n+1 \\
        &= -(-n-1) \\
        &= -\lfloor{-x}\rfloor
    \end{align*}
    Therefore, $\lceil{x}\rceil = - \lfloor{-x}\rfloor$ for all real numbers $x$.
\end{proof}

\begin{problem}{4}
    Let $a$, $b$, and $c$ be arbitrary integers and let $r = a \bmod b$. Prove that if $c \mid a$ and $c \mid b$, then $c \mid r$.
\end{problem}
\begin{proof}
    We are given that $a$, $b$, and $c$ are arbitrary integers, and that $r = a \bmod b$. \\
    We need to prove that $c \mid r$ given that $c \mid a$ and $c \mid b$. \\
    By definition of mod, $r = a \bmod b$ means that $a = bq + r$ for some integer $q$. \\
    $c \mid a$ can be written as $a = ck_1$ for some integer $k_1$. \\
    $c \mid b$ can be written as $b = ck_2$ for some integer $k_2$. \\ \\
    Substituting these into the equation $a = bq + r$:
    \begin{align*}
        a &= bq + r \\
        ck_1 &= (ck_2)q + r \\
        r &= ck_1 - (ck_2)q \\
        r &= c(k_1 - k_2q)
    \end{align*}
    Since $k_1 - k_2q$ is an integer since it is the sum of products of integers, this means that $r$ is an integer multiple of $c$. By definition, $c$ divides $r$. \\
    Therefore, $c \mid r$.
\end{proof}

\begin{problem}{5}
    Prove that for all integers $n > 10$, $n^2 - 100$ is composite.
\end{problem}
\begin{proof}
    We need to prove that $n^2 - 100$ is composite for all integers $n > 10$. \\
    By definition of a composite number, $n^2 - 100$ must be a positive integer that has at least one positive divisor other than 1 and itself.
    \begin{equation*}
        n^2 - 100 = (n - 10)(n + 10)
    \end{equation*}
    For $n > 10$ and $n \neq 11$:
    \begin{align*}
        (n-10) &> 1 \\
        (n+10) &> n > 1
    \end{align*}
    Since $n^2-100$ can be written as the product of two integers greater than $1$, $(n-10)$ and $(n+10)$, it is composite. \\ \\
    For $n = 11$:
    \begin{align*}
        (n-10) &= 1 \\
        (n+10) &= 21 \\
        n^2 - 100 &= 21 \\
        &= 1 \cdot 21 \\
        &= 3 \cdot 7
    \end{align*}
    When $n = 11$, since $n^2 - 100$ can be written as the product of two integers other than 1 and itself ($21$), it is composite. \\ \\
    Thus, for $n > 10$, $n^2 - 100$ is always composite.
\end{proof}

\begin{problem}{6}
    Prove that for all integers $n$, if $3 \nmid n$, then $3 \mid (n^2 +2)$.\ (\textit{Hint:} Quotient-Remainder Theorem will help immensely with this.)
\end{problem}
\begin{proof}
    We need to prove that for all integers $n$, if $3 \nmid n$, then $3 \mid (n^2 + 2)$. \\
    By Quotient-Remainder Theorem, for any integer $n$ and a positive integer $d$, there exist unique integers $q$ and $r$ such that:
    \begin{align*}
        n &= dq + r \\
        0 &\leq r < d
    \end{align*}
    For $d = 3$, and since $3 \nmid n$, the possible remainders for $n$ are $r = 1$ or $r = 2$. \\ \\
    \textbf{Case 1:} If $r = 1$, then $n = 3q + 1$ for some integer $q$. Squaring both sides:
    \begin{align*}
        n^2 &= {(3q + 1)}^2 \\
        &= 9q^2 + 6q + 1 \\
        n^2 + 2 &= 9q^2 + 6q + 3 \\
        &= 3(3q^2 + 2q + 1)
    \end{align*}
    \textbf{Case 2:} If $r = 2$, then $n = 3q + 2$. Squaring both sides:
    \begin{align*}
        n^2 &= {(3q + 2)}^2 \\
        &= 9q^2 + 12q + 4 \\
        n^2 + 2 &= 9q^2 + 12q + 6 \\
        &= 3(3q^2 + 4q + 2)
    \end{align*}
    In both cases, $3$ divides $n^2 + 2$, thus proving that for all integers $n$, if $3 \nmid n$, then $3 \mid (n^2 +2)$.
\end{proof}

% ------------------------------------------------------------------------------
\end{document}
% ------------------------------------------------------------------------------