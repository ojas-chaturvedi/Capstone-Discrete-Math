\section{Sequences, Mathematical Induction, and Recursion}
% 
\subsection{Sequences}
% 
\subsubsection{Definition}
A function whose domain is either all of the integers between two given integers (\textbf{finite sequence})
\begin{align*}
    a_m, a_{m+1}, a_{m+2}, \dots, a_n
\end{align*}
or all of the integers greater than or equal to a given integer (\textbf{infinite sequence})
\begin{align*}
    a_m, a_{m+1}, a_{m+2}, \dots
\end{align*}
Each individual element $a_k$ (read ``$a$ sub $k$'') is called a \textbf{term}. The $k$ in $a_k$ is called a \textbf{subscript} or \textbf{index}.
An \textbf{explicit formula} or \textbf{general formula} for a sequence is a rule that shows how the values of $a_k$ depend on $k$.
\subsubsection{Alternating Sequence}
A sequence whose terms alternate in sign.
\begin{align*}
    a_1, -a_2, a_3, -a_4, \dots
\end{align*}
Example:
\begin{align*}
    c_j &= (-1)^j & \text{for all integers $j \geq 0$ }
\end{align*}
This sequence will oscillate endlessly between $1$ and $-1$.
% 
\newpage
\subsubsection{Finding an Explicit Formula to fit Given Initial Terms}
Note: Any such formula is a guess, but it is very useful to be able to make such guesses.
\begin{problem}
    Find an explicit formula for a sequence that has the following initial terms:
    \begin{align*}
        1, - \frac{1}{4}, \frac{1}{9}, - \frac{1}{16}, \frac{1}{25}, - \frac{1}{36}, \ldots
    \end{align*}
\end{problem}
\begin{proof}
    Denote the general term of the sequence by $a_k$ and suppose the first term of the sequence is $a_1$. Then observe that the denominator of each term is a perfect square. Thus, the terms can be rewritten as
    \begin{align*}
        \frac{1}{1^2}, - \frac{1}{2^2}, \frac{1}{3^2}, - \frac{1}{4^2}, \frac{1}{5^2}, - \frac{1}{6^2}, \ldots \\
        a_1, a_2, a_3, a_4, a_5, a_6, \ldots
    \end{align*}
    Note that the denominator of each term equals the square of the subscript of that term, and that the numerator equals $\pm 1$. Hence
    \begin{align*}
        a_k &= \frac{\pm 1}{k^2}
    \end{align*}
    The $\pm 1$ oscillates back and forth between $+1$ and $-1$, meaning $a_k$ is an alternating sequence. In this specific example, when $k$ is odd, the numerator is $+1$ and when $k$ is even, the numerator is $-1$. Therefore, the explicit formula of $a_k$ can be written as
    \begin{align*}
        a_k &= \frac{(-1)^{k+1}}{k^2} & \text{for all integers $k \geq 1$}
    \end{align*}
    If we were to make the first term $a_0$ instead of $a_1$, the formula would be
    \begin{align*}
        a_k &= \frac{(-1)^k}{(k+1)^2} & \text{for all integers $k \geq 0$}
    \end{align*}
\end{proof}

\subsubsection{Summation Notation}

\subsection{Mathematical Induction I}


\subsection{Mathematical Induction II}


\subsection{Strong Mathematical Induction and the Well-Ordering \\ Principle for the Integers}


\subsection{Application: Correctness of Algorithms}


\subsection{Defining Sequences Recursively}


\subsection{Solving Recurrence Relations by Iteration}


\subsection{Second-Order Linear Homogeneous Recurrence Relations with Constant Coefficients}


\subsection{General Recursive Definitions and Structural Induction}

