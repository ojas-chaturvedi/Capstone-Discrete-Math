\documentclass[11pt]{article}
\usepackage{amsmath,amssymb,amsthm, float}
\usepackage{graphicx} \graphicspath{ {./images/} } % for images
\usepackage[margin=1in]{geometry} % for page dimensions
\usepackage{fancyhdr, enumitem, mathrsfs} % for nice headers, enumerated lists, calligraphic letters
\setlength{\parindent}{0pt}
\setlength{\parskip}{5pt plus 1pt}
\setlength{\headheight}{25.2842pt}
\newcommand\question[2]{\vspace{.25in}\hrule{\textbf{#1}: #2}\vspace{.5em}\hrule\vspace{.10in}}
\renewcommand\part[1]{\vspace{.10in}\textbf{(#1)}}
\newcommand{\todo}{\fbox{TO-DO}\ \ } % for TODOs
% \newcommand\algorithm{\vspace{.10in}\textbf{Algorithm: }}
% \newcommand\correctness{\vspace{.10in}\textbf{Correctness: }}
% \newcommand\runtime{\vspace{.10in}\textbf{Running time: }}
\pagestyle{fancyplain}
\lhead{{\NAME} \\ \EMAILID}
\chead{\textbf{Study Guide for Test \TESTNUM}}
\rhead{Capstone: Discrete Math \\ Test: December 25, 2023}
\begin{document}{\raggedright}
%Section A==============Change the values below to match your information==================
\newcommand\NAME{Ojas Chaturvedi}  % your name
\newcommand\EMAILID{oj.chaturvedi.2024@gmail.com}
\newcommand\TESTNUM{3}

\section*{Big Ideas}
\begin{itemize}
    \item \textbf{Induction} | be familiar with establishing \textit{base case} and \textit{inductive step}.
    \begin{itemize}
        \item Base Case: $P(0)$ is true.
        \item Inductive Step: Assume $P(n)$ for a \underline{specific} $n$ and show $P(n) \rightarrow P(n+1)$.
    \end{itemize}
    \item \textbf{Strong Induction} | It's like normal induction, except we potentially assume multiple base cases and need to assume $P(0)$ through $P(n)$ are true to show that $P(n+1)$ is true.
    \item \textbf{Well-Ordering Principle} | If $S$ is a set of one or more integers all greater than some fixed integers, then $S$ has a least element.
    \item \textbf{Induction Application: Algorithms} | You can show an algorithm is ``correct'' by applying induction to variables that go through iterations of a loop. This involves use of a \textit{loop invariant}, a predicate that is true before the loop and remains true after passing through the loop, and the \textit{guard}, which is the name given to the predicate condition that keeps statements in the loop. We denote $I(n)$ to be the $n$th iteration of the loop invariant through the loop.
    \begin{itemize}
        \item Basis Step | Identical to base case. Check that the pre-condition is true before the loop, e.e. pre-condition implies $I(0)$ is true.
        \item Induction Step | Show that if loop invariant $I(n)$ and guard $G$ are true before the loop, then $I(n+1)$ is true after the loop.
        \item Eventual Falsity of the Guard | Show that eventually $G$ will become false (we don't want the loop to run infinitely!).
        \item Correctness of Post-Condition | Check that the post-condition is true after the loop, i.e. if $N$ is the smallest value for which $I(N)$ is true and the guard $G$ becomes false, then the post-condition is also true.
    \end{itemize}
    I'm aware this looks like a lot, but realistically it is identical to induction with the added steps of checking the loop doesn't run infinitely and that the end result isn't contradictory.
    \item \textbf{Recurrence Relations} | A recurrence relation is defined as a formula for a sequence that defines elements in terms of previous elements. For example:
    \begin{align*}
        a_n &= a_{n-1} + 3a_{n-2} - 7a_{n-3}
    \end{align*}
    relates values in the sequence to the previous three values.
    \item \textbf{Solving Second-Order Recurrence Relations} | For a recurrence relation of the form $a_n = Aa_{n-1} + Ba_{n-2}$, you can find sequences that satisfy the relation by solving the characteristic equation
    \begin{align*}
        t^2 - At -B = 0
    \end{align*}
    The solution sequence would then be $a_n = t^n$ for $n \geq 0$. \\
    For second-order recurrence relations with initial conditions $a_0$ and $a_1$, you can find an explicit formula by solving for constants $C$ and $D$ as follows:
    \begin{itemize}
        \item If $r$ and $s$ are \textit{unique} roots of the characteristic equation:
        \begin{align*}
            a_n = Cr^n + Ds^n
        \end{align*}
        \item If $r$ is a \textit{repeated} roots of the characteristic equation:
        \begin{align*}
            a_n = Cr^n + Dnr^n
        \end{align*}
    \end{itemize}
    \item \textbf{Structural Induction} | Sets are defined recursively using a BASE group of elements, a list of RECURSION rules to create new elements, and a RESTRICTION that these are the only ways to form elements of the set. You can perform induction on the set to check if a condition is true by applying the recursion rules of the set to the base case of the induction.
    \item \textbf{Element Argument of Sets} | To show that $X \subseteq Y$, show that if you assume $x \in X$ arbitrarily, that $x \in Y$ as well.
    \item \textbf{Set Equality} | To show that sets $X = Y$, you must show $X \subseteq Y$ and $Y \subseteq X$.
    \item \textbf{Element Method for the Empty Set} | To show a set $X = \emptyset$, use proof by contradiction - assume $\exists x \in X$ and demonstrate this as a contradiction.
    \item \textbf{Power Sets} | The set of all subsets of a set $X$ is called the \textit{power set} of $X$ and is denoted $\mathscr{P}(X)$.
\end{itemize}

\section*{Practice Textbook Problems}
\textbf{5.2:} 11, 16, 33-35 \\ 
\textbf{5.3:} 8, 11, 13, 19, 23, 26, 29 \\ 
\textbf{5.4:} 8-11, 17, 22, 23, 26, 32 \\ 
\textbf{5.5:} 1-5 \\ 
\textbf{5.6:} 3-5, 9-12, 17 \\ 
\textbf{5.7:} 1, 3-6 \\ 
\textbf{5.8:} 11-16, 20, 24 \\ 
\textbf{5.9:} 4-11 \\ 
\textbf{6.1:} 3-5, 12, 13, 22, 24, 27, 30, 31 \\ 
\textbf{6.2:} 7-9, 11-13, 16, 25-29, 38, 40 \\ 

\end{document}