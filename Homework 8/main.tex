\documentclass[name=Ojas\ Chaturvedi, emailid=oj.chaturvedi.2024, course=Capstone:\ Discrete\ Math, num=8, deadline={November\ 2,\ 2023}]{homework}

\usepackage{hw-shortcuts}

\begin{document}

\problem{4.8.15}
Use the Euclidean algorithm to hand-calculate the greatest common divisors of 832 and 10933.
\begin{proof}
    By the Euclidean algorithm,
    \begin{align*}
        10933 \Mod {832} &= 117 \\
        832 \Mod {117} &= 13 \\
        117 \Mod {13} &= 0
    \end{align*}
    By Lemma 4.8.2, $\gcd(10933,832) = \gcd(832,117) = \gcd(117,13) = \gcd(13,0)$. \\
    By Lemma 4.8.1, $\gcd(13,0) = 13$. \\
    Thus, $\gcd(10933,832) = 13$.
\end{proof}

\problem{4.8.16}
Use the Euclidean algorithm to hand-calculate the greatest common divisors of 4131 and 2431.
\begin{proof}
    By the Euclidean algorithm,
    \begin{align*}
        4131 \Mod {2431} &= 1700 \\
        2431 \Mod {1700} &= 731 \\
        1700 \Mod {731} &= 238 \\
        731 \Mod {238} &= 17 \\
        238 \Mod {17} &= 0
    \end{align*}
    By Lemma 4.8.2, $\gcd(4131,2431) = \gcd(2431,1700) = \gcd(1700,731) = \gcd(731,238) = \gcd(238,17) = \gcd(17,0)$. \\
    By Lemma 4.8.1, $\gcd(17,0) = 17$. \\
    Thus, $\gcd(4131,2431) = 17$.
\end{proof}

\problem{4.8.19}
Prove that for all positive integers $a$ and $b$, $a | b$ if, and only if, $\gcd(a,b) = a$. (Note that to prove ``$A$ if, and only if, $B$,'' you need to prove ``if $A$ then $B$'' and ``if $B$ then $A$.'')
\begin{proof}
    First, we need to prove that if $a | b$, then $\gcd(a, b) = a$. \\
    Since $a|b$, $b = ac$, with some integer $c$. \\
    We  know that $b\geq a$. \\
    Using QRT, we can rewrite $b$ as $aq + r$, with some integers $q, r$ and $0\leq r < a$. \\
    By Lemma 4.8.2, $\gcd(b, a) = \gcd(a, r)$. However, since $a | b$, $r = 0$. \\
    Thus, $\gcd(a, b) = \gcd(a, 0) = a$, proving the conditional statement. \\
    Next, we need to prove that if $\gcd(a, b) = a$, then $a | b$. \\
    By definition of gcd, if $\gcd(a, b) = a$, then $a | a$ and $a | b$, proving the conditional statement. \\
    As both conditionals have been proven, it is true that for all positive integers $a, b$, $a | b$ if, and only if, $\gcd(a, b) = a$.
\end{proof}

\problem{4.8.20}
\begin{enumerate}
    \item[a.] Prove that if $a$ and $b$ are integers, not both zero, and $d = \gcd(a,b)$, then $a/d$ and $b/d$ are integers with no common divisor that is greater than one.
    \item[b.] Write an algorithm that accepts the numerator and denominator of a fraction as input and produces as output the numerator and denominator of that fraction written in lowest terms. (The algorithm may call upon the Euclidean algorithm as needed.)
\end{enumerate}
\begin{proof}
    \begin{enumerate}
        \item[a.]
            \textbf{Proof by contradicton:} \\
            If $a, b$ are non-zero integers, and $d = \gcd(a, b)$, then $a/d$ and $b/d$ are integers with a common divisor greater than one. \\
            This leads to a contradiction since $a/d$ and $b/d$ have a common divisor greater than one, meaning $d\neq \gcd(a, b)$ by definition of GCD. \\
            Therefore, it is true that if $a, b$ are integers, not both zero, and $d = \gcd(a, b)$, then $a/d$ and $b/d$ are integers with no common divisor greater than one.
        \item[b.]
            Let the Euclidean algorithm be equivalent to euc(a, b) in the following code: \\
            def frac\_simplify(num, dem): \\
                gcd = euclidean\_algorithm(num, dem) \\
                new\_num = num/gcd \\
                new\_dem = dem/gcd \\
                return (new\_num, new\_dem)
    \end{enumerate}
\end{proof}

\problem{4.8.27}
Prove that for all positive integers $a$ and $b$, $a | b$ if, and only if, $\lcm(a,b) = b$. \\ \\
Definition: The least common multiple of two nonzero integers $a$ and $b$, denoted $\lcm(a,b)$, is the positive integer $c$ such that
\begin{enumerate}
    \item[a.] $a|c$ and $b|c$
    \item[b.] for all positive integers $m$, if $a|m$ and $b|m$, then $c \leq m$.
\end{enumerate}
\begin{proof}
    First, we need to prove that if $a | b$, then $\lcm(a, b) = b$. As $a | b$, $b = ac$, with some integer $c$. We also know that $b\geq a$. As the LCM of $a, b$ must be divisible by both of these integers, the LCM has to be $\geq b$. Let's test if $b$ works as the LCM. Since $b | b$, $a | b$, and there isn't an integer $m$ less than $b$ s.t. $b | m$ and $a | m$, it is proven that $\lcm(a, b) = b$, proving the conditional statement. Next, we need to prove that if $\lcm(a, b) = b$, then $a | b$. By definition of lcm, if $\lcm(a, b) = b$, then $a | b$ and $b | b$, proving the conditional statement. Therefore, it is true that for all positive integers $a, b$, $a | b$ if, and only if, $\lcm(a, b) = b$.
\end{proof}

\separator

\problem{5.1.44}
Write using summation or product notation:
\begin{align*}
    \left(1^3-1\right)-\left(2^3-1\right)+\left(3^3-1\right)-\left(4^3-1\right)+\left(5^3-1\right)
\end{align*}
\begin{proof}
    \begin{align*}
        \sum_{n=1}^{5} ((n^3-1) \cdot (-1)^{n-1})
    \end{align*}
\end{proof}

\problem{5.1.45}
Write using summation or product notation:
\begin{align*}
    \left(2^2-1\right) \cdot\left(3^2-1\right) \cdot\left(4^2-1\right)
\end{align*}
\begin{proof}
    \begin{align*}
        \prod_{n=2}^{4} (n^2 -1)
    \end{align*}
\end{proof}

\problem{5.1.46}
Write using summation or product notation:
\begin{align*}
    \frac{2}{3 \cdot 4}-\frac{3}{4 \cdot 5}+\frac{4}{5 \cdot 6}-\frac{5}{6 \cdot 7}+\frac{6}{7 \cdot 8}
\end{align*}
\begin{proof}
    \begin{align*}
        \sum_{n=2}^{6} \left ( \left (\frac{n}{(n+1) \cdot (n+2)} \right) \cdot (-1)^{n} \right)
    \end{align*}
\end{proof}

\problem{5.1.77}
\begin{enumerate}
    \item[a.] Prove that $n!+2$ is divisible by $2$, for all integers $n \geq 2$.
    \item[b.] Prove that $n!+k$ is divisible by $k$, for all integers $n \geq 2$ and $k=2,3, \ldots, n$.
\end{enumerate}
\begin{proof}
    \begin{enumerate}
        \item[a.]
            Let $n$ be any integer greater than or equal to $2$. \\
            By definition of factorial,
            \begin{align*}
                n! = n \cdot (n-1) \cdots 2 \cdot 1 
            \end{align*}
            Since $2$ will always be a factor of $n!$, $n!$ can be written as $2k$ for some integer $k$.
            Thus,
            \begin{align*}
                n! + 2 &= 2k + 2 \\
                &= 2(k + 1)
            \end{align*}
            $k + 1$ is an integer, so $n! + 2$ can be written as $2m$ for some integer $m$. \\
            Therefore, $n! + 2$ is divisible by $2$ by definition of divisibility.
        \item[b.]
            Let $n$ be any integer greater than or equal to $2$. \\
            Let $k$ be any integer greater than or equal to $2$ and less than or equal to $n$. \\
            By definition of factorial, $n! = n \cdot (n-1) \cdots 2 \cdot 1$, and since $2 \leq k \leq n$, $k$ will always be a factor of $n!$. \\
            \begin{align*}
                n! + k &= n \cdot (n-1) \cdots 2 \cdot 1 + k \\
                &= n \cdot (n-1) \cdots (k+1) \cdot k \cdot (k-1) \cdots 2 \cdot 1 + k \\
                &= k \cdot (n \cdot (n-1) \cdots (k+1) \cdot (k-1) \cdots 2 \cdot 1 + 1)
            \end{align*}
            Therefore, $n! + k$ can be written as $k \cdot m$ for some integer $m = (n \cdot (n-1) \cdots (k+1) \cdot (k-1) \cdots 2 \cdot 1 + 1)$.
            Thus, $n! + k$ is divisible by $k$ by definition of divisibility.
    \end{enumerate}
\end{proof}

\problem{5.1.79}
Prove that if $p$ is a prime number and $r$ is an integer with $0<r<p$, then $p \choose r$ is divisible by $p$.
\begin{proof}
    Let $p$ be any prime number and $r$ be any integer such that $0 < r < p$.
    By definition of $p \choose r$,
    \begin{align*}
        p \choose r &= \frac{p!}{r! \cdot (p-r)!}
    \end{align*}
    By definition of factorial,
    \begin{align*}
        p! &= p \cdot (p-1) \cdots 2 \cdot 1
    \end{align*}
    Therefore, $p!$ can be written as $p \cdot n$ for some integer $n = (p-1) \cdots 2 \cdot 1$
    Putting $p!$ back into the equation for $p \choose r$,
    \begin{align*}
        p \choose r &= \frac{p \cdot n}{r! \cdot (p-r)!}
    \end{align*}
    Therefore, $p \choose r$ can be written as $p \cdot m$ for some integer $m = \frac{n}{r! \cdot (p-r)!}$. \\
    Thus, $p \choose r$ is divisible by $p$ by definition of divisibility.
\end{proof}

\separator

\problem{5.2.7}
Prove the statement using mathematical induction. \\
For all integers $n \geq 1$,
\begin{align*}
    1 + 6 + 11 + 16 + \cdots + (5n-4) = \frac{n(5n-3)}{2}
\end{align*}
\begin{proof}
    Let the property $P(n)$ be the equation:
    \begin{align*}
        1 + 6 + 11 + 16 + \cdots + (5n-4) = \frac{n(5n-3)}{2}
    \end{align*}
    First, we must prove that $P(1)$ is true (basis step).
    \begin{align*}
        1 &= \frac{1(5 \cdot 1-3)}{2}
    \end{align*}
    The left-hand side of the equation is $1$, and the right-hand side of the equation is
    \begin{align*}
        \frac{1(5 \cdot 1-3)}{2} &= \frac{1(5-3)}{2} \\
        &= \frac{1(2)}{2} \\
        &= 1
    \end{align*}
    Thus, $P(1)$ is true. \\
    Now, suppose that $P(k)$ is true for some integer $k \geq 1$ (inductive hypothesis). That is,
    \begin{align*}
        1 + 6 + 11 + 16 + \cdots + (5k-4) = \frac{k(5k-3)}{2}
    \end{align*}
    We must show that $P(k+1)$ is true (inductive step). That is,
    \begin{align*}
        1 + 6 + 11 + 16 + \cdots + (5(k+1)-4) &= \frac{(k+1)(5(k+1)-3)}{2} \\
        1 + 6 + 11 + 16 + \cdots + (5k-1) &= \frac{(k+1)(5k+2)}{2}
    \end{align*}
    The left-hand side of $P(k+1)$ is
    \begin{align*}
        1 + 6 + 11 + 16 + \cdots + (5k-1) & = 1 + 6 + 11 + 16 + \cdots + (5k-4) + (5k-1) \\
        & \quad \text{by making the next-to-last term explicit} \\
        &= \frac{k(5k-3)}{2} + (5k-1) \\
        & \quad \text{by substitution of the inductive hypothesis} \\
        &= \frac{5k^2-3k}{2} + (5k-1) \\
        & \quad \text{by distribution} \\
        &= \frac{5k^2-3k+10k-2}{2} \\
        & \quad \text{by combining like terms} \\
        &= \frac{5k^2+7k-2}{2} \\
        & \quad \text{by combining like terms}
    \end{align*}
    The right-hand side of $P(k+1)$ is
    \begin{align*}
        \frac{(k+1)(5k+2)}{2} &= \frac{5k^2+7k+2}{2} \\
        & \quad \text{by distribution} \\
        &= \frac{5k^2+7k-2}{2} \\
        & \quad \text{by combining like terms}
    \end{align*}
    Thus, the left-hand side of $P(k+1)$ is equal to the right-hand side of $P(k+1)$, and $P(k+1)$ is true. \\
    Since we have proved both the basis step and the inductive step, we conclude that the statement is true using mathematical induction.
\end{proof}

\problem{5.2.9}
Prove the statement using mathematical induction. \\
For all integers $n \geq 3$,
\begin{align*}
    4^3 + 4^4 + 4^5 + \cdots + 4^n = \frac{4(4^n-16)}{3}
\end{align*}
\begin{proof}
    Let $P(n)$ be the equation:
    \begin{align*}
        4^3 + 4^4 + 4^5 + \cdots + 4^n = \frac{4(4^n-16)}{3}
    \end{align*}
    First, we must prove that $P(3)$ is true (basis step).
    \begin{align*}
        4^3 &= \frac{4(4^3-16)}{3}
    \end{align*}
    The left-hand side of the equation is $64$, and the right-hand side of the equation is
    \begin{align*}
        \frac{4(4^3-16)}{3} &= \frac{4(64-16)}{3} \\
        &= \frac{4(48)}{3} \\
        &= \frac{192}{3} \\
        &= 64
    \end{align*}
    Thus, $P(3)$ is true. \\
    Now, suppose that $P(k)$ is true for some integer $k \geq 3$ (inductive hypothesis). That is,
    \begin{align*}
        4^3 + 4^4 + 4^5 + \cdots + 4^k = \frac{4(4^k-16)}{3}
    \end{align*}
    We must show that $P(k+1)$ is true (inductive step). That is,
    \begin{align*}
        4^3 + 4^4 + 4^5 + \cdots + 4^{k+1} = \frac{4(4^{k+1}-16)}{3}
    \end{align*}
    The left-hand side of $P(k+1)$ is
    \begin{align*}
        4^3 + 4^4 + 4^5 + \cdots + 4^{k+1} &= 4^3 + 4^4 + 4^5 + \cdots + 4^k + 4^{k+1} \\
        & \quad \text{by making the next-to-last term explicit} \\
        &= \frac{4(4^k-16)}{3} + 4^{k+1} \\
        & \quad \text{by substitution of the inductive hypothesis} \\
        &= \frac{4(4^k-16) + 3(4^{k+1})}{3} \\
        & \quad \text{by combining like terms} \\
        &= \frac{4 \cdot 4^k - 64 + 3(4 \cdot 4^k)}{3} \\
        & \quad \text{by distribution} \\
        &= \frac{4 \cdot 4^k - 64 + 12 \cdot 4^k}{3} \\
        & \quad \text{by distribution} \\
        &= \frac{16 \cdot 4^k - 64}{3} \\
        & \quad \text{by combining like terms} \\
        &= \frac{4 \cdot (4 \cdot 4^k) - 64}{3} \\
        & \quad \text{by factoring} \\
        &= \frac{4 ((4 \cdot 4^k) - 16)}{3} \\
        & \quad \text{by factoring} \\
        &= \frac{4 (4^{k+1} - 16)}{3} \\
        & \quad \text{by exponent product rule}
    \end{align*}
    Thus, the left-hand side of $P(k+1)$ is equal to the right-hand side of $P(k+1)$, and $P(k+1)$ is true. \\
    Since we have proved both the basis step and the inductive step, we conclude that the statement is true using mathematical induction.
\end{proof}

\problem{5.2.12}
Prove the statement using mathematical induction.
\begin{align*}
    \frac{1}{1 \cdot 2} + \frac{1}{2 \cdot 3} + \cdots + \frac{1}{n(n+1)} = \frac{n}{n+1}
\end{align*}
for all integers $n \geq 1$.
\begin{proof}
    Let the property $P(n)$ be the equation:
    \begin{align*}
        \frac{1}{1 \cdot 2} + \frac{1}{2 \cdot 3} + \cdots + \frac{1}{n(n+1)} = \frac{n}{n+1}
    \end{align*}
    First, we must prove that $P(1)$ is true (basis step). \\
    \begin{align*}
        \frac{1}{1 \cdot 2} &= \frac{1}{1+1} \\
        \frac{1}{2} &= \frac{1}{2}
    \end{align*}
    Thus, $P(1)$ is true. \\
    Now, suppose that $P(k)$ is true for some integer $k \geq 1$ (inductive hypothesis). That is,
    \begin{align*}
        \frac{1}{1 \cdot 2} + \frac{1}{2 \cdot 3} + \cdots + \frac{1}{k(k+1)} = \frac{k}{k+1}
    \end{align*}
    We must show that $P(k+1)$ is true (inductive step). That is,
    \begin{align*}
        \frac{1}{1 \cdot 2} + \frac{1}{2 \cdot 3} + \cdots + \frac{1}{(k+1)((k+1)+1)} &= \frac{k+1}{(k+1)+1} \\
        \frac{1}{1 \cdot 2} + \frac{1}{2 \cdot 3} + \cdots + \frac{1}{(k+1)(k+2)} &= \frac{k+1}{k+2} \\
        \frac{1}{1 \cdot 2} + \frac{1}{2 \cdot 3} + \cdots + \frac{1}{k^2 + 3k + 2} &= \frac{k+1}{k+2}
    \end{align*}
    The left-hand side of $P(k+1)$ is
    \begin{align*}
        \frac{1}{1 \cdot 2} + \frac{1}{2 \cdot 3} + \cdots + \frac{1}{k^2 + 3k + 2} &= \frac{1}{1 \cdot 2} + \frac{1}{2 \cdot 3} + \cdots + \frac{1}{k(k+1)} + \frac{1}{k^2 + 3k + 2} \\
        & \quad \text{by making the next-to-last term explicit} \\
        &= \frac{k}{k+1} + \frac{1}{k^2 + 3k + 2} \\
        & \quad \text{by substitution of the inductive hypothesis} \\
        &= \frac{k(k+2)+1}{k^2+3k+2} \\
        & \quad \text{by combining like terms} \\
        &= \frac{k^2 +2k + 1}{k^2 + 3k + 2} \\
        & \quad \text{by distribution} \\
        &= \frac{(k+1)(k+1)}{(k+1)(k+2)} \\
        & \quad \text{by factoring}\\
        &= \frac{k+1}{k+2} \\
        & \quad \text{by cancelling out like terms}
    \end{align*}
    Thus, the left-hand side of $P(k+1)$ is equal to the right-hand side of $P(k+1)$, and $P(k+1)$ is true. \\
    Since we have proved both the basis step and the inductive step, we conclude that the statement is true using mathematical induction.
\end{proof}

\problem{5.2.17}
Prove the statement using mathematical induction.
\begin{align*}
    \prod_{i=0}^{n} \left(\frac{1}{2i+1} \cdot \frac{1}{2i+2} \right) = \frac{1}{(2n+2)!}
\end{align*}
for all integers $n \geq 0$.
\begin{proof}
    Let the property $P(n)$ be the equation:
    \begin{align*}
        \prod_{i=0}^{n} \left(\frac{1}{2i+1} \cdot \frac{1}{2i+2} \right) = \frac{1}{(2n+2)!}
    \end{align*}
    First, we must prove that $P(0)$ is true (basis step). \\
    \begin{align*}
        \prod_{i=0}^{0} \left(\frac{1}{2i+1} \cdot \frac{1}{2i+2} \right) &= \frac{1}{(2 \cdot 0+2)!} \\
        \left(\frac{1}{2 \cdot 0+1} \cdot \frac{1}{2 \cdot 0+2} \right) &= \frac{1}{2!} \\
        \left(\frac{1}{1} \cdot \frac{1}{2} \right) &= \frac{1}{2} \\
        \frac{1}{2} &= \frac{1}{2}
    \end{align*}
    Thus, $P(0)$ is true. \\
    Now, suppose that $P(k)$ is true for some integer $k \geq 0$ (inductive hypothesis). That is,
    \begin{align*}
        \prod_{i=0}^{k} \left(\frac{1}{2i+1} \cdot \frac{1}{2i+2} \right) = \frac{1}{(2k+2)!}
    \end{align*}
    We must show that $P(k+1)$ is true (inductive step). That is,
    \begin{align*}
        \prod_{i=0}^{k+1} \left(\frac{1}{2i+1} \cdot \frac{1}{2i+2} \right) &= \frac{1}{(2(k+1)+2)!} \\
        \prod_{i=0}^{k+1} \left(\frac{1}{2i+1} \cdot \frac{1}{2i+2} \right) &= \frac{1}{(2k+4)!}
    \end{align*}
    The left-hand side of $P(k+1)$ is
    \begin{align*}
        \prod_{i=0}^{k+1} \left(\frac{1}{2i+1} \cdot \frac{1}{2i+2} \right) &= \prod_{i=0}^{k} \left(\frac{1}{2i+1} \cdot \frac{1}{2i+2} \right) \cdot \left(\frac{1}{2(k+1)+1} \cdot \frac{1}{2(k+1)+2} \right) \\
        & \quad \text{by writing the $(k+1)$st term separately from the first $k$ terms} \\
        &= \frac{1}{(2k+2)!} \cdot \left(\frac{1}{2(k+1)+1} \cdot \frac{1}{2(k+1)+2} \right) \\
        & \quad \text{by substitution of the inductive hypothesis} \\
        &= \frac{1}{(2k+2)!} \cdot \left(\frac{1}{2k+3} \cdot \frac{1}{2k+4} \right) \\
        & \quad \text{by distribution} \\
        &= \frac{1}{(2k+4) \cdot (2k+3) \cdot (2k+2)!} \\
        &= \frac{1}{(2k+4)!} \\
        & \quad \text{by definition of factorial}
    \end{align*}
    Thus, the left-hand side of $P(k+1)$ is equal to the right-hand side of $P(k+1)$, and $P(k+1)$ is true. \\
    Since we have proved both the basis step and the inductive step, we conclude that the statement is true using mathematical induction.
\end{proof}

\problem{5.2.19}
Use mathematical induction, the product rule from calculus, and the facts that $\frac{d(x)}{dx} = 1$ and that $x^{k+1} = x \cdot x^k$ to prove that for all integers $n \geq 1$, $\frac{d(x^n)}{dx} = nx^{n-1}$.
\begin{proof}
    Let the property $P(n)$ be the equation:
    \begin{align*}
        \frac{d(x^n)}{dx} = nx^{n-1}
    \end{align*}
    First, we must prove that $P(1)$ is true (basis step). \\
    \begin{align*}
        \frac{d(x^1)}{dx} &= 1 \cdot x^{1-1} \\
        \frac{d(x)}{dx} &= 1 \cdot x^{0} \\
        1 &= 1
    \end{align*}
    Thus, $P(1)$ is true. \\
    Now, suppose that $P(k)$ is true for some integer $k \geq 1$ (inductive hypothesis). That is,
    \begin{align*}
        \frac{d(x^k)}{dx} = kx^{k-1}
    \end{align*}
    We must show that $P(k+1)$ is true (inductive step). That is,
    \begin{align*}
        \frac{d(x^{k+1})}{dx} &= (k+1) \cdot x^{(k+1)-1} \\
        \frac{d(x^{k+1})}{dx} &= (k+1) \cdot x^{k}
    \end{align*}
    The left-hand side of $P(k+1)$ is
    \begin{align*}
        \frac{d(x^{k+1})}{dx} &=\frac{d}{dx}(x^k \cdot x) \\
        & \quad \text{by power product rule} \\
        &= x \cdot \frac{d}{dx}x^k + x^k \cdot \frac{d}{dx}x \\
        & \quad \text{by derivative product rule} \\
        &= x \cdot k(x^{k-1}) + x^k \cdot 1\\
        & \quad \text{by derivatives} \\
        &= k(x^k) + x^k \\
        & \quad \text{by combining like terms} \\
        &= (k+1) \cdot x^k \\
        & \quad \text{by factoring}
    \end{align*}
    Thus, the left-hand side of $P(k+1)$ is equal to the right-hand side of $P(k+1)$, and $P(k+1)$ is true. \\
    Since we have proved both the basis step and the inductive step, we conclude that the statement is true using mathematical induction.
\end{proof}

\separator

\problem{5.3.9}
Prove the statement using mathematical induction: $7^n - 1$ is divisible by $6$, for each integer $n \geq 0$.
\begin{proof}
    Let the property $P(n)$ be the sentence:
    \begin{align*}
        7^n - 1 \text{ is divisible by } 6
    \end{align*}
    First, let's prove that $P(0)$ is true (basis step).
    \begin{align*}
        7^0 - 1 &= 1 -1 \\
        &= 0 \\
        &= 6 \cdot 0
    \end{align*}
    Thus, $P(0)$ is true. \\
    Now, suppose that $P(k)$ is true for some integer $k \geq 0$ (inductive hypothesis). That is,
    \begin{align*}
        7^k - 1 \text{ is divisible by } 6
    \end{align*}
    By the definition of divisibility, for some integer $r$,
    \begin{align*}
        7^k - 1 &= 6r
    \end{align*}
    We must show that $P(k+1)$ is true (inductive step). That is,
    \begin{align*}
        7^{k+1} - 1 \text{ is divisible by } 6
    \end{align*}
    The left-hand side of $P(k+1)$ is:
    \begin{align*}
        7^{k+1} - 1 &= 7 \cdot 7^k - 1 \\
        & \quad \text{by the product exponent rule} \\
        &= 7 \cdot (6r + 1) - 1 \\
        & \quad \text{by definition of $7^k$} \\
        &= 42r + 7 - 1 \\
        & \quad \text{by distribution} \\
        &= 42r + 6 \\
        &= 6(7r + 1)
    \end{align*}
    $7r + 1$ is an integer since it is the sum of products of integers, so $7^{k+1} - 1$ can be written as $6m$ for some integer $m = (7r + 1)$. \\
    By the definition of divisibility, $7^{k+1} - 1$ is divisible by $6$, and thus, $P(k+1)$ is true. \\
    Since we have proved both the basis step and the inductive step, we conclude that the statement is true using mathematical induction.
\end{proof}

\problem{5.3.10}
Prove the statement using mathematical induction: $n^3 -7n +3$ is divisible by $3$, for each integer $n \geq 0$.
\begin{proof}
    Let the property $P(n)$ be the sentence:
    \begin{align*}
        n^3 -7n +3 \text{ is divisible by } 3
    \end{align*}
    First, we must prove that $P(0)$ is true (basis step).
    \begin{align*}
        0^3 -7 \cdot 0 +3 \text{ is divisible by } 3
    \end{align*}
    \begin{align*}
        0^3 -7(0) +3 &= 0 - 0 + 3 \\
        &= 3 \\
        &= 3 \cdot 1
    \end{align*}
    Thus, $P(0)$ is true. \\
    Now, suppose that $P(k)$ is true for some integer $k \geq 0$ (inductive hypothesis). That is,
    \begin{align*}
        k^3 -7k +3 \text{ is divisible by } 3
    \end{align*}
    By the definition of divisibility, for some integer $r$,
    \begin{align*}
        k^3 -7k +3 &= 3r
    \end{align*}
    We must show that $P(k+1)$ is true (inductive step). That is,
    \begin{align*}
        (k+1)^3 -7(k+1) +3 \text{ is divisible by } 3
    \end{align*}
    The left-hand side of $P(k+1)$ is:
    \begin{align*}
        (k+1)^3 -7(k+1) +3 &= (k+1)(k+1)(k+1) -7(k+1) +3 \\
        & \quad \text{by expanding} \\
        &= (k+1)(k^2 + 2k + 1) -7(k+1) +3 \\
        & \quad \text{by the product exponent rule} \\
        &= k^3 + 2k^2 + k + k^2 + 2k + 1 -7k -7 +3 \\
        & \quad \text{by expanding} \\
        &= k^3 + 3k^2 -4k -3 \\
        & \quad \text{by combining like terms} \\
        &= 3r + 7k - 3 + 3k^2 - 4k -3 \\
        & \quad \text{by substitution of the inductive hypothesis} \\
        &= 3r + 3k^2 + 3k - 6 \\
        & \quad \text{by combining like terms} \\
        &= 3(r + k^2 + k - 2) \\
        & \quad \text{by factoring}
    \end{align*}
    $r + k^2 + k - 2$ is an integer since it is a sum of the product of integers, so $(k+1)^3 -7(k+1) +3$ can be written as $3m$ for some integer $m = (r + k^2 + k - 2)$. \\
    By the definition of divisibility, $(k+1)^3 -7(k+1) +3$ is divisible by $3$, and thus, $P(k+1)$ is true. \\
    Since we have proved both the basis step and the inductive step, we conclude that the statement is true using mathematical induction.
\end{proof}

\problem{5.3.17}
Prove the statement using mathematical induction: $1 + 3n \leq 4^n$, for every integer $n \geq 0$.
\begin{proof}
    Let the property $P(n)$ be the inequality:
    \begin{align*}
        1 + 3n \leq 4^n
    \end{align*}
    First, we must prove that $P(0)$ is true (basis step).
    \begin{align*}
        1 + 3(0) &\leq 4^0 \\
        1 &\leq 1
    \end{align*}
    Thus, $P(0)$ is true. \\
    Now, suppose that $P(k)$ is true for some integer $k \geq 0$ (inductive hypothesis). That is,
    \begin{align*}
        1 + 3k \leq 4^k
    \end{align*}
    We must show that $P(k+1)$ is true (inductive step). That is,
    \begin{align*}
        1 + 3(k+1) &\leq 4^{k+1} \\
        1 + 3k+3 &\leq 4^{k+1} \\
        3k+4 &\leq 4^{k+1} \\
    \end{align*}
    The left-hand side of $P(k+1)$ is:
    \begin{align*}
        3k+4 &= 3k+1+3 \\
        &\leq 4^k + 3 \\
        & \quad \text{by substitution of the inductive hypothesis} \\
        &\leq 4^k + 4^1 \\
        & \quad \text{since $3 < 4^1$} \\
        &\leq 4^{k+1}
    \end{align*}
    Thus, the left-hand side of $P(k+1)$ is less than or equal to the right-hand side of $P(k+1)$, and $P(k+1)$ is true. \\
    Since we have proved both the basis step and the inductive step, we conclude that the statement is true using mathematical induction.
\end{proof}

\problem{5.3.20}
Prove the statement using mathematical induction: $2^n < (n+2)!$, for every integer $n \geq 0$.
\begin{proof}
    Let the property $P(n)$ be the inequality:
    \begin{align*}
        2^n < (n+2)!
    \end{align*}
    First, we must prove that $P(0)$ is true (basis step).
    \begin{align*}
        2^0 &< (0+2)! \\
        1 &< 2
    \end{align*}
    Thus, $P(0)$ is true. \\
    Now, suppose that $P(k)$ is true for some integer $k \geq 0$ (inductive hypothesis). That is,
    \begin{align*}
        2^k < (k+2)!
    \end{align*}
    We must show that $P(k+1)$ is true (inductive step). That is,
    \begin{align*}
        2^{k+1} &< ((k+1)+2)! \\
        2^{k+1} &< (k+3)!
    \end{align*}
    The left-hand side of $P(k+1)$ is:
    \begin{align*}
        2^{k+1} &= 2 \cdot 2^k \\
        &< 2 \cdot (k+2)! \\
        & \quad \text{by substitution of the inductive hypothesis} \\
        &< (k+3)(k+2)! \\
        & \quad \text{since $2 < k+3$} \\
        &< (k+3)!
    \end{align*}
    Thus, the left-hand side of $P(k+1)$ is less than the right-hand side of $P(k+1)$, and $P(k+1)$ is true. \\
    Since we have proved both the basis step and the inductive step, we conclude that the statement is true using mathematical induction.
\end{proof}

\problem{5.3.36}
In a round-robin tournament each team plays every other team exactly once. If the teams are labeled $T_1, T_2, \ldots, T_n,$ then the the outcome of such a tournament can be represented by a drawing, called a directed graph, in which the teams are represented as dots and an arrow is drawn from one dot to another if, and only if, the team represented by the first dot beats the team represented by the second dot. For example, the directed graph below shows one outcome of a round-robin tournament involving five teams, A, B, C, D,and E. \\ \\
Use mathematical induction to show that in any round-robin tournament involving $n$ teams, where $n \geq 2$, it is possible to label the teams $T_1, T_2, \ldots, T_n$ so that $T_i$ beats $T_{i+1}$ for all $i = 1, 2, \ldots, n-1$. For instance, one such labeling in the example above is $T_1 = A, T_2 = B, T_3 = C, T_4 = E, T_5 = D$. (Hint: Given $k+1$ teams, pick one -- say $T'$ beats $T_1$, $T'$ loses to the first $m$ teams (where $1 \leq m \leq k-1$) and beats the $(m+1)$st team, and $T'$ loses to all the other teams.)
\begin{proof}
    Problem confuses me, will ask for help either in class or in student hours. However, I know that its a proof by induction for the property of a sequence.
\end{proof}

\end{document}